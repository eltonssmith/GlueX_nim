%=======================+=========================
%================  Controls  ================
%=================================================

\section[Slow controls (Hovanes)]{Slow controls \label{sec:controls}}
\subsection{Architecture \label{sec:controlsarchitechture}}
As any modern sophisticated particle detector system \gx{} requires being able to monitor 
and control tens of thousands of different variables that define the state of the experimental hardware. Different variable values need to be acquired, displayed, archived, used as inputs to control loops continually with a high degree of reliability. The GlueX slow control system consists of three layers. The first layer consists of the remote units such as high voltage or low voltage power chassis, magnet power supplies, temperature controller, LabView applications, Programmable Logic Controller or PLC-based applications, which directly interact with the hardware and contain almost all of the controls loops. The second layer is the Supervisory Control and Data Acquisition (SCADA) layer which is implemented in the form of EPICS\footnote{https://epics.anl.gov} Input/Output Controllers or (IOC). This layer provides the the interface between the low level application and the higher level applications using EPICS ChannelAccess protocol. The highest level, referred as Experiment Control System (ECS), contains the application such as Human-Machine Interface (HMI), the alarm system and data archiving. This structure allows for relatively easy and seamless addition and integration of new components into the overall controls system of \gx{}. 

\subsection{Remote Units \label{sec:controlsinterface}}
\gx{} uses a large number of over-the-counter units that provide us with controls over the hardware used in the experiment. For instance most of the detector high voltages are provided by boards CAEN SYx527 voltage chassis\footnote{https://www.caen.it/subfamilies/mainframes/} while the low and bias voltages are provided by boards residing in Wiener MPOD chassis\footnote{http://www.wiener-d.com/sc/power-supplies/mpod--lvhv/mpod-crate.html}. These two types of system provide most of the voltage with the exception of tagger microscope and forward calorimeter ( see Sec.~\ref{sec:fcal} ) where custom made systems were developed which provide voltage regulation and interact with EPICS-based layer through higher level interfaces using custom-made protocols.  
There are various beamline devices that need to be moved during beam operations. We use stepper motors to move these motorized stages via Newport XPS universal multi-axis motion controller\footnote{https://www.newport.com/c/xps-universal-multi-axis-motion-controller} that allows for execution of complex trajectories involving multiple axis. All of the stage referencing, motion profile computations and encoder-based closed-loop control occur within the controller chassis after the basic parameters such as positions and velocities are provided by the user via TCP/IP based interface to EPICS.   

Often when installing complex systems, such as a superconducting magnet, we needed to develop a custom controls system with a large number of input, outputs channels and sophisticated logic suited for the particular system. For those cases we utilized Allen-Bradley PLC \footnote{https://ab.rockwellautomation.com} systems. These systems are designed for industrial system that allows for a modular design and provide reliability and require minimal maintenance. All of the controls loops are programmed within the PLC application. These types of PLC-s are interfaced with EPICS through TCP/IP EtherNet/IP proprietary protocol to allow access to the process variables from the higher level applications.  
\subsection{SCADA layer \label{sec:archiver}}
\subsection{Experiment Control System \label{sec:alarms}}
