%=======================+=========================
%================  Controls  ================
%=================================================

\section[Slow controls (Hovanes)]{Slow controls \label{sec:controls}}
\subsection{Architecture \label{sec:controlsarchitechture}}
As any modern sophisticated particle detector system \gx{} requires being able to monitor 
and control tens of thousands of different variables that define the state of the experimental hardware. Different variable values need to be acquired, displayed, archived, used as inputs to control loops continually with a high degree of reliability. The GlueX slow control system consists of three layers. The first layer consists of the remote units such as high voltage or low voltage power chassis, magnet power supplies, temperature controller, LabView applications, Programmable Logic Controller or PLC-based applications, which directly interact with the hardware and contain almost all of the control loops. The second layer is the Supervisory Control and Data Acquisition (SCADA) layer which is implemented in the form of EPICS\footnote{https://epics.anl.gov.} Input/Output Controllers or (IOC's). This layer provides the the interface between the low level application and the higher level applications using EPICS ChannelAccess protocol. The highest level, referred as Experiment Control System (ECS), contains the application such as Human-Machine Interface (HMI), the alarm system and data archiving. This structure allows for relatively easy and seamless addition and integration of new components into the overall controls system of \gx{}.    

\subsection{Remote Units \label{sec:controlsinterface}}
\gx{} uses a large number of commercial units that provide us with control over the hardware used in the experiment. For instance most of the detector high voltages are provided by the CAEN SYx527 voltage mainframe\footnote{https://www.caen.it/subfamilies/mainframes/.} while the low and bias voltages are provided by boards residing in Wiener MPOD chassis\footnote{http://www.wiener-d.com/sc/power-supplies/mpod--lvhv/mpod-crate.html.}. These two system  types provide most of the voltage with the exception of tagger microscope and forward calorimeter ( see Sec.~\ref{sec:fcal} ) where custom systems were developed which provide voltage regulation and interact with EPICS-based layer through higher level interfaces using custom protocols.  
There are various beam line devices that need to be moved during beam operations. We use stepper motors to move these motorized stages via Newport XPS universal multi-axis motion controller\footnote{https://www.newport.com/c/xps-universal-multi-axis-motion-controller.} that allows for execution of complex trajectories involving multiple axes. All of the stage referencing, motion profile computations and encoder-based closed-loop control occur within the controller chassis after the basic parameters such as positions and velocities are provided by the user via TCP/IP based interface to EPICS.   

Often when installing complex systems, such as a superconducting magnet, we needed to develop a custom controls system with a large number of input, outputs channels and sophisticated logic suited for the particular system. For those cases we utilized Allen-Bradley PLC \footnote{https://ab.rockwellautomation.com.} systems. These systems are designed for industrial systems that allow for a modular design and provide reliability and require minimal maintenance. All of the controls loops are programmed within the PLC application. These types of PLC-s are interfaced with EPICS through TCP/IP EtherNet/IP proprietary protocol to allow access to the process variables from the higher level applications.  

\subsection{Supervisory Control and Data Acquisition layer \label{sec:archiver}}
The Supervisory Control and Data Acquisition (SCADA) layer is the middle layer that distributes the process variables allowing the higher level, and sometimes lower level, applications to use various process variables of Hall D controls system. This layer is based on EPICS and uses  ChannelAccess protocol to publish the values of the variables. And because the accelerator controls also uses EPICS this allows us to efficiently exchange the information between the experiment and the accelerator operations. There are a few dozens of software Input/Out Controller (IOC) processes running on the hosts in the experiment control room that collect the data from different components of the lowest layer. Each of these IOC-s is configured to be able to communicate using the protocol appropriate for the remote units from which it needs to to exchange data. For instance the EPICS IOC controlling the voltage for the FDC detector needs to be able to communicate with Wiener MPOD and CAEN SYx527 voltage chassis. 
\subsection{Experiment Control System \label{sec:alarms}}
