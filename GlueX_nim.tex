%\documentclass[final,5p,times,twocolumn]{elsarticle}
%\documentclass[3pd]{elsarticle}
\documentclass{elsarticle}
 % The package hyperref provides LaTeX the ability to create hyperlinks within the document. Comment out the next line to load document onto the arXiv.
\usepackage{hyperref}
\usepackage{xcolor}

% The package hyphenat allows the hyphenation of compound words containing a dash
%\usepackage{hyphenat}
% Further attempts to properly hyphenate words containing dashes
%\usepackage[shortcuts]{extdash}

%\usepackage{draftwatermark}

% Use the option doublespacing or reviewcopy to obtain double line spacing
% \documentclass[doublespacing]{elsart}

%\documentclass[<options>]{elsarticle}
%where the options can be the following:
%(1) preprint ? default option which formats the document for submission to
%Elsevier journals. 
%(2) review ? similar to preprint option, but increases the baselineskip to facilitate easier review process.
%(3) 1p ? formats the article to the look and feel of the final format of model 1+ journals. This is always single column style.
%(4) 3p ? formats the article to the look and feel of the final format of model 3+ journals. If the journal is a two column model, use twocolumn option in combination.
%(5) 5p ? formats for model 5+ journals. This is always two column style.
% etc., see http://www.elsevier.com/framework_authors/misc/elsdoc.pdf

% if you use PostScript figures in your article
% use the graphics package for simple commands      
%\usepackage{graphics}
% or use the graphicx package for more complicated commands
\usepackage{graphicx,lscape,rotating}

% The amssymb package provides various useful mathematical symbols
\usepackage{amssymb}
% The amsmath package provides miscellaneous enhancements for improving the information structure and printed output of documents that contain mathematical formulas.
\usepackage{amsmath}
% The amsthm package provides an enhanced version of LATEX?s \newtheorem command for defining theorem-like environments. 
%\usepackage{amsthm}
\usepackage{multirow}
\newcommand\EatDot[1]{}

% The package lineno adds line numbers to LaTeX documents.
\usepackage{lineno}

\begin{document}  

\input{symbols.tex}
\linenumbers   
 
\begin{frontmatter} 

% Title, authors and addresses   

\title{The \gx~ Beamline and Detector}

\input{authors.tex}



\begin{abstract}
The \gx~ experiment at Jefferson Lab has been designed to study photoproduction reactions with a 9-GeV linearly polarized photon beam. The energy and arrival time of beam photons are tagged using a scintillator
hodoscope and a scintillating fiber array. The photon flux is determined using a pair spectrometer, while the linear polarization of the photon beam is determined using a polarimeter based on triplet photoproduction.
Charged-particle tracks from interactions in the central target are analyzed in a solenoidal field using a central straw-tube
drift chamber and six packages of planar chambers with cathode strips and drift wires. Electromagnetic showers are reconstructed in a cylindrical scintillating fiber calorimeter inside the magnet and
a lead-glass array downstream. Charged particle identification is achieved by measuring energy loss in the wire chambers and using the flight time of particles between the target and detectors outside the magnet. The signals from all detectors are recorded with flash ADCs and/or pipeline TDCs into memories allowing trigger decisions with a latency of 3.3\,$\mu$s. The detector operates routinely at
trigger rates of 40~kHz and data rates of 600 megabytes per second. We describe the photon beam, the \gx~ detector components, electronics, data-acquisition and monitoring systems, and the performance of the experiment during the first three 
years of operation.
\end{abstract}   

%\begin{keyword}
% keywords here, in the form: keyword \sep keyword
% electromagnetic calorimeter \sep sampling calorimeter \sep scintillating fibers  \sep silicon photomultipliers \sep MPPC \sep GlueX
% PACS codes here, in the form: \PACS code \sep code
% \PACS 29.40.Mc \sep 29.40.Vj 
% \end{keyword}

\end{frontmatter}
  
  
   

%===============================+================================
%============    Project Background  and Detector Requirements =============
%================================================================

% \newpage 

\tableofcontents

%=======================+=========================
%================  Introduction  ================
%=================================================
\section[The \gx{} experiment]{\label{sec:gluexexperiment} The \gx{} experiment}
The search for Quantum ChromoDynamics (QCD) exotics has been ongoing for several decades and uses data from a wide range of experiments and production mechanisms. Historically, such searches have looked for the gluonic excitations of mesons, searching for both states of pure glue, glueballs, and hybrid mesons where the gluonic field binding the quark-anti-quark pair has been excited. Reviews of searches for glueballs~\cite{Crede:2008vw} show that most of these looked for scalar glueballs, where the searches rely on over population of nonets as well as unusual meson decay patterns. In the search for hybrid mesons~\cite{Meyer:2010ku,Meyer:2015eta}, efforts have focused on particles with non-quark-anti-quark (exotic) quantum numbers where good evidence exists for an isospin $1$ state, the $\pi_{1}(1600)$. Looking collectively at past studies, we find that data from high-statistics photoproduction experiments in the energy regime above $6$~GeV is lacking. 

\begin{figure}[h!]\centering
\includegraphics[width=0.75\textwidth]{figures/GlueX-graphic.jpg}
\caption[]{\label{fig:gluex_cut-away}(Color online)A cut-away drawing of the \GX{} detector in Hall D, not to scale.}
\end{figure}
The \emph{Glu}onic \emph{Ex}citation (\gx{}) experiment at the 
US Department of Energy's Thomas Jefferson National Accelerator Facility (JLab) \cite{jlab-ref} has been built to both search for and map out the spectrum of exotic hybrid mesons using a high-energy (9~GeV) linearly-polarized photon beam incident on a proton target\cite{gluex-ref}. The detector is nearly hermetic for both charged particles and photons, allowing for reconstruction of exclusive final states. A 2-T solenoidal magnet surrounds the drift-chamber based tracking. Two electromagnetic calorimeters cover the central and forward regions, and a scintillation detector downstream provides particle-identification capability through time-of-flight measurements. The \gx{} detector and beamline are shown schematically in Figure~\ref{fig:gluex_cut-away}.


\subsection[The Hall-D complex]{The Hall-D complex \label{sec:gluexexperiment:complex}}
The \gx{} experiment is housed in the Hall-D complex at JLab (see Fig.\ref{fig:CEBAF-graphic}). This new facility starts with an extracted electron beam at the north end of the Continuous Electron Beam Accelerator Facility (CEBAF) \cite{Leemann:2001dg}. These are the highest-energy electrons at the lab, up to 12~GeV, due to an extra one-half pass of acceleration they have relative to three other experimental halls (A, B and C). The electron beam enters the Tagger Hall where it produces linearly polarized photons through coherent bremsstrahlung on a 50~$\mu$m thick diamond crystal radiator.
The scattered electrons pass through a tagger magnet and are bent into tagging detectors. A high-resolution scintillating-fiber array covers the 8 to 9~GeV energy range, and a tagger hodoscope covers photon energies both from 9~GeV to the endpoint, and from 8~GeV to 3~GeV. Those electrons not interacting in the diamond are directed into a 60 kW electron beam dump. The tagged photons pass through a tunnel to the Hall-D experimental hall. The distance from the radiator to the 5~mm diameter primary collimator, which removes off-axis incoherent photons, is 75 m. The front face of the collimator is instrumented with an active collimator to aid in beam tuning.  The beamline and tagging system are described in Section\,\ref{sec:beamline}.

Downstream of the primary collimator is a thin beryllium radiator used by both the Triplet Polarimeter, which measures the linear polarization of the photons, and a Pair Spectrometer, which is used to measure the flux of the photons. More information on the production, tagging and monitoring of the photon beam can be found in Section~\ref{sec:beamline}. The photons then pass through a 30~cm long liquid hydrogen target at the heart of the \gx{} detector. Photons which did not interact in the target travel to the end of the experimental hall where they enter the photon beam dump.

\begin{figure}[tbp]\centering
\includegraphics[width=0.75\textwidth]{figures/CEBAF-graphic.png}
\caption[]{\label{fig:CEBAF-graphic}(Color online) Schematic of the CEBAF accelerator showing the additions made during the 12-GeV project. The Hall D complex is located at the north-east end.}
\end{figure}

The \gx{} detector is based on a 4~m long solenoidal magnet which is operated at a maximum field of 2~T. The layout of the detector is shown in Fig.~\ref{fig:layout_spectrometer}, while more information on the solenoid can be found in Section~\ref{sec:solenoid}. The liquid-hydrogen target is located 65~cm inside the upstream bore of the magnet. It consists of a 2~cm diameter, 29.25~cm long volume of hydrogen and is described in Section~\ref{sec:target}. Surrounding the target is the Start Counter which consists of 30 thin scintillator paddles that bend to a nose on the down-stream end of the target. The Start Counter is the primary detector that identifies the radio-frequency (RF) bunch that contained the incident electron. More information can be found in Section~\ref{sec:scintillators}. 

% ======================================================================================

\begin{figure*}[tbp]
\centering
  \includegraphics[angle=0,viewport=95 115 628 500,clip,width=1.0\linewidth]{figures/gluex_spectrometer_drawing_01_bw}%
  \caption[layout]{GlueX spectrometer layout. Dimensions are given in mm. The
    numbers show the Z-coordinates of the detectors' centers, or of
    the front face of the calorimeter modules in case of the FCAL.
    Glossary: 
              SC  - Start Counter (Section \ref{sec:st}), 
              CDC - Central Drift Chamber (Section \ref{sec:cdc}), 
              FDC - Forward Drift Chamber (Section \ref{sec:fdc}),
              BCAL - Barrel Calorimeter (Section \ref{sec:bcal}), 
              TOF -  Time-of-Flight hodoscope (Section \ref{sec:tof}), 
              FCAL - Forward Calorimeter (Section \ref{sec:fcal}).
%
%    \begin{tabular}{lll}
%       Name  & Detector & Section \\ \hline
%              SC  & Start Counter & \ref{sec:st} \\ 
%              CDC & Central Drift Chamber  & \ref{sec:cdc} \\ 
%              FDC & Forward Drift Chamber  & \ref{sec:fdc} \\ 
%              BCAL & Barrel Calorimeter    & \ref{sec:bcal} \\ 
%              TOF &  Time-of-Flight hodoscope & \ref{sec:tof} \\ 
%              FCAL & Forward Calorimeter    & \ref{sec:fcal} \\ 
%    \end{tabular}
    \label{fig:layout_spectrometer}
  }
\end{figure*}

% ======================================================================================


%\begin{figure}[htbp]\centering
%\includegraphics[width=0.7\textwidth]{figures/GlueX_Sketch.pdf}  
%\caption{\label{fig:gluexsketch} (Color online)         
%Sketch of GlueX detector.  The main systems of the detector are the Start %Counter \cite{Pooser:2019rhu}, the Central Drift Chamber (CDC) %\cite{VanHaarlem:2010yq} the Forward Drift Chamber (FDC) \cite{Pentchev2017281}, %a scintillator-based Time of Flight (TOF) wall and a lead-glass Forward %Calorimeter (FCAL) \cite{MORIYA201360}. The Barrel Calorimeter (BCAL) is %sandwiched between the drift chambers and the inner radius of the solenoid.}   
%\end{figure}

Starting at a radius of 10~cm from the beam line is the Central Drift Chamber, a cylindrical straw-tube detector. The active volume of the chamber extends from 48~cm upstream to 102~cm downstream of the target center, and from 10~cm to 56~cm in radius. It consists of 28 layers of straw tubes in both axial and two stereo orientations. Downstream of the central tracker is the Forward Drift Chamber. This device consists of four packages, each containing 6 planar layers in alternating $u$-$y$-$v$ orientations. Both cathodes and anodes are read out from the forward chambers, providing 3D space point measurements. More details on the tracking system can be found in Sections~\ref{sec:tracking} and \ref{sec:trackingperformance}. 

Downstream of the magnet is the Time-of-Flight wall. The detector consists of two layers of scintillator paddles in a crossed pattern, and in conjunction with the Start Counter, is used to measure the flight-time of charged particles. More information can be found in Section~\ref{sec:scintillators}. 
Produced photons in \gx{} are detected by two calorimeter systems. The Barrel Calorimeter, located inside the solenoid, consists of layers of scintillating fibers alternating with lead sheets. The Forward Calorimeter is downstream of the Time-of-Flight wall, and consists of $2800$ lead-glass blocks. More information can be found in Section~\ref{sec:calorimeters}.

\subsection[Experimental Requirements]{Experimental Requirements \label{sec:intro:requirements}}
In order to reconstruct exclusive final states, the \gx{} detector must be able to reconstruct both charged particles, $\pi^{\pm}$, $K^{\pm}$ and $p/\bar{p}$, and particles decaying into photons, $\pi^{\circ}$, $\eta$, $\omega$ and $\eta^{\prime}$. For this, the charged particles and photons must be reconstructed with good momentum and energy resolution. The experiment must also be able to reconstruct the incident photon's energy (8 to 9~GeV) with high accuracy ($0.1$\%) and know its degree of linear polarization (40\%) at the percent-level. Finally, many of the interesting final states involve more than five particles. Thus, the \gx{} detector must also be nearly hermetic for both charged particles and photons, with an acceptance that is reasonably uniform, well understood, and accurately modeled in simulation.

The typical momentum resolution for charged particles is $1$--$3\%$, while for very-forward high-momentum particles it is somewhat worse at around $8$-$9\%$. For most charged particles, the tracking system has nearly hermetic acceptance for polar angles in the lab from about $1^\circ-2^{\circ}$ to $150^{\circ}$. Due to the target thickness, protons with momentum below about 250~MeV/c are not detected, and pions with momentum under 200~MeV/c can have spiraling trajectories in the detector, making reconstruction challenging. $dE/dx$ measured in the Central Drift Chamber can separate pions and protons up to about 800~MeV/c, while time-of-flight can separate forward-going pions and kaons up to about 2~GeV/c.

For photons, the typical energy resolution is 5 to 6\%$/\sqrt{E_{\gamma}}$. There is some variation in the Barrel Calorimeter resolution, depending on the incident angle of the photon, but generally, photons above about 60~MeV can be detected. The interaction point along the beam direction is determined by comparing the information from the readouts on the upstream and downstream ends of the detector. The Forward Calorimeter can reconstruct photons whose energy is larger than 100~MeV, with uniform resolution across the face of the detector. There is a gap region between the calorimeters at around $11^{\circ}$, where energy can be lost due to leakage. Both photon detection efficiency and energy resolution are degraded in this region. 
 
\subsection{Data Requirements \label{sec:intro:data_requirements}}
To be able to carry out the physics analyses in small bins of energy and momentum transfer, the detector not only must have the ability to reconstruct exclusive final states but also collect sufficient statistics. While exact cross sections are not known, it is expected that those of interest will be in the 10~nb range, with the largest cross sections of interest around 1~$\mu$b. The initial phase of the \GX{} experiment has run with a data acquisition system capable of collecting data using photon beams of a few $10^{7}~\gamma/$s in the coherent peak (8.4-9 GeV), while the expectation is to run about 2.5 times higher rates in the future. %\textcolor{blue}{(Do we want to mention the total photon flux above some cutoff energy? This may be better in the beamline section.)}
The data acquisition system ran routinely at 40\,kHz with raw event sizes of $15-20$ kilobytes, collecting about 600~megabytes of data per second. With trigger improvements, future running is expected at about 90 kHz and about 1~gigabyte per second. Details of the trigger and data acquisition can be found in Sections~\ref{sec:trig} and \ref{sec:daq}.

\subsection{Coordinate system \label{sec:intro:coordinates}}
The experimental area is located 
off the north east corner of the accelerator. The z-axis is defined along the nominal beamline increasing downstream (toward the east). The coordinate system 
is right-handed with the y-axis pointing vertically up and the x-axis pointing approximately north. 
The origin is located 50.8\,cm (20 inches) downstream of the upstream side of the upstream endplate of the solenoid. That places the nominal center of the target at (0,0,65\,cm).

\input{beamline}
\input{solenoid}
% \clearpage   % avoid formatting problems with empty sections
% \input{detector_overview}
\section[Target]{Target \label{sec:target} }
A schematic diagram of the \gx{} liquid hydrogen cryotarget is shown in Fig.~\ref{fig:Target}. The major components of the system are a pulse tube cryocooler,\footnote{Cryomech model PT415.} a condenser, and a target cell.  These items are contained within an aluminum and stainless steel `L'-shaped vacuum chamber with an extension of closed-cell foam\footnote{Rohacell 110XT, Evonik Industries AG.} surrounding the target cell. In turn, the \gx{} Start Counter (Sec.~\ref{sec:st}) surrounds the foam chamber and is supported by the horizontal portion of the vacuum chamber. Polyimide foils, 100~$\mu$m thick, are used at the upstream and downstream ends of the chamber as beam entrance and exit windows. The entire system, including the control electronics, vacuum pumps, gas-handling system, and tanks for hydrogen storage, is mounted on a small cart that is attached to a set of rails for insertion into the \gx{} solenoid.  To satisfy flammable gas safety requirements, the system is connected at multiple points to a nitrogen-purged ventilation pipe that extends outside Hall D.
\begin{figure*}
\begin{center}
\includegraphics[width=4.5in]{figures/TargetSchematic3.pdf}
\end{center}
\caption{Simplified process and instrumentation diagram for the GlueX liquid hydrogen target (not to scale).
In the real system, the P-trap is above the level of the target cell and is used to
promote convective cooling of the target cell from room temperature.}
\label{fig:Target}
\end{figure*}

Hydrogen gas is stored inside two 200~l tanks and
is cooled and condensed into a small copper and stainless steel container,
the condenser, that is thermally anchored to the second cooling stage of the cryocooler. 
The first stage of the cryocooler is used to
cool the H$_2$ gas to about 50~K before it enters the condenser.
The first stage also cools a copper thermal shield that surrounds all
lower-temperature components of the system except for the
target cell itself, which is wrapped in a few layers of aluminized-mylar/cerex insulation.

The condenser is comprised of a copper C101 base
sealed to a stainless steel can with an indium O-ring.  Numerous vertical 
fins are cut into the copper base, giving a large surface area for condensing hydrogen gas.
A heater and a pair of calibrated Cernox thermometers\footnote{Cernox, Lake Shore Cryotronics.}
are attached outside the condenser, and are used to regulate the heater temperature when the
system is filled with liquid hydrogen.

The target cell, shown in Fig.~\ref{fig:TargetCell}, is similar to
designs used in Hall B at JLab~\cite{HAKOBYAN2008218}.  
The cell walls are made from 100-$\mu$m-thick aluminized
polyimide sheet wrapped in a conical shape and glued along the edge,
overlapping into a 2~mm wide scarf joint.  
The conical shape prevents bubbles from collecting inside the cell, while the
scarf joint reduces the stress riser at the glue joint.  This conical
tube is glued to an aluminum base, 
along with stainless steel fill and return tubes leading to the condenser, a feed-through for two calibrated Cernox thermometers inside the cell, and a
polyamide-imide support for the reentrant upstream beam window.  
Both the upstream and downstream beam
windows are made of non-aluminized,
100~$\mu$m thick polyimide films that have been extruded into the
shapes indicated in Fig.~\ref{fig:TargetCell}. These windows are clearly
visible in Fig.~\ref{fig:z-vertex} where reconstructed vertex positions are shown. All items are glued together using
a two-part epoxy\footnote{3M Scotch-Weld epoxy adhesive DP190 Gray.}
that has been in reliable use at cryogenic temperatures for long periods. 
A second  heater, attached to the aluminum base,
is used to empty the cell for background measurements.
The base is attached to a kinematic mount, which is in turn
supported inside the vacuum chamber using a system of carbon fiber rods.    
The mount is used to correct the pitch and yaw
of the cell, while $X$, $Y$, and $Z$ adjustments 
are accomplished using positioning screws on the target cart. 


During normal operation, a sufficient amount of hydrogen gas is condensed from the storage tanks
until the target cell, condenser, and interconnecting piping are filled with liquid hydrogen
and an equilibrium pressure of about 19~psia is achieved.  
The condenser temperature is regulated at 18~K, while the
liquid in the cell cools to about 20.1~K. The latter temperature is 1~K below the saturation
temperature of H$_2$, which eliminates boiling within the cell and permits a more
accurate determination of the fluid density, 
$71.2 \pm 0.3$~mg/cm$^3$.  
The system can be cooled from room temperature and filled with liquid hydrogen in
approximately six hours.  Prior to measurements using an empty target cell, the liquid hydrogen is boiled back into the storage tanks in about five minutes.  H$_2$ gas continues to condense and drain towards the target cell, but the condensed hydrogen is immediately 
evaporated by the cell heater.  In this way, the cell does not warm above 40~K and
can be re-filled with liquid hydrogen in about twenty minutes.

\begin{figure}
\includegraphics[width=3.5in]{figures/GluexCell_mm.pdf}
\caption{Target cell for the liquid hydrogen target.  Dimensions are in mm.  }
\label{fig:TargetCell}
\end{figure}

Operation of the cryotarget is highly automated, requires minimal user intervention,
and has operated in a very reliable and predictable manner throughout the
experiment. 
The target controls\footnote{The control logic uses National Instruments CompactRIO 9030.} are handled by a LabVIEW program, 
while a standard EPICS softIOC running in Linux provides a
bridge between the controller and JLab's EPICS enviroment (see Section\,\ref{sec:controls}).     
Temperature readback and control of the condenser and target cell thermometers
are managed by a four-input temperature
controller\footnote{Lake Shore Model 336.} with PID control loops of 50 and 100~W.
Strain gauge pressure sensors measure the fill and return pressures with 0.25\% 
accuracy.  When filled with subcooled liquid, 
the long-term temperature ($\pm 0.2$~K) and pressure ($\pm 0.1$~psi)
stability of the liquid hydrogen enable a determination of the density to better than 0.5\%.



\section{Tracking detectors \label{sec:tracking}}
\subsection[Central drift chamber]{Central drift chamber \label{sec:cdc}}

The Central Drift Chamber (CDC) is a cylindrical straw-tube drift chamber which is used to track charged particles by providing position, timing and energy loss measurements~\cite{VanHaarlem:2010yq,GlueXCDCNIM}.
The CDC is situated inside the Barrel Calorimeter, surrounding the target and Start Counter. 
The active volume of the CDC is traversed
by particles coming from the hydrogen target with polar angles between $6^{\circ}$ and $168^{\circ}$, with optimum 
coverage for polar angles between $29^{\circ}$ and $132^{\circ}$.  
The CDC contains 3522 anode wires of 20~$\mu$m diameter gold-plated tungsten inside Mylar\footnote{www.mylar.com} straw tubes of diameter 1.6~cm in $28$ layers,
located in a cylindrical volume which is 1.5~m long, with an inner radius of 10~cm and outer radius of 56~cm, as measured from the beamline.  
Readout is from the upstream end. 
Fig.\,\ref{fig:CDC_schematic} shows a schematic diagram of the detector.

\begin{figure}[tbp]
\begin{center}
\includegraphics[width=0.7\textwidth]{figures/CDC_schematic.pdf}  
\caption{\label{fig:CDC_schematic}          
  Cross-section through the cylindrically symmetric Central Drift Chamber, along the beamline.}  
\end{center}
\end{figure}

The straw tubes are arranged in 28 layers; 12 layers are axial, and 16 layers are at stereo angles of $\pm 6^{\circ}$ to provide position information along the beam direction.
The stereo angle was chosen to balance the extra tracking information provided by the unique combination of stereo and axial straws along a trajectory against the size of the unused volume inside the chamber at each transition between stereo and axial layers. 
Fig.\,\ref{fig:CDC_stereotubes} shows the CDC during construction. 

\begin{figure}[tbp]
\begin{center}
\includegraphics[width=0.7\textwidth]{figures/CDC_stereotubes.jpg}  
\caption{\label{fig:CDC_stereotubes}          
  The Central Drift Chamber during construction. A partially completed layer of stereo straw tubes is shown, surrounding a layer of straw tubes at the opposite stereo angle. Part of the carbon fiber endplate, two temporary tension rods and some of the 12 permanent support rods linking the two endplates can also be seen.}  
\end{center}
\end{figure}

The volume surrounding the straws is enclosed by an inner cylindrical wall of 0.5\,mm G10 fiberglass, an outer cylindrical wall of 1.6\,mm aluminum, and two circular endplates. 
The upstream endplate is made of aluminum, while the downstream endplate is made of carbon fiber. The endplates are connected by 12 aluminum support rods. 
Holes milled through the endplates support the ends of the straw tubes, which were glued into place using several small components per tube, described more fully in~\cite{GlueXCDCNIM}.  
These components also support the anode wires, which were installed with 30~g tension.
At the upstream end, these components are made of aluminum and were glued in place using conductive epoxy\footnote{TIGA 920-H, www.loctite.com}. 
This attachment method provides a good electrical connection to the inside walls of the straw tubes, which are coated in aluminum.
The components at the downstream end are made of Noryl plastic\footnote{www.sabic.com} and were glued in place using conventional non-conductive epoxy\footnote{3M Scotch-Weld DP460NS, www.3m.com}.
The materials used for the downstream end were chosen to be as lightweight as feasible so as to minimize the energy loss of charged particles passing through them. 

At each end of the chamber, a cylindrical gas plenum is located outside the endplate.  
The gas supply runs in 12 tubes through the volume surrounding the straws into the downstream plenum. 
There the gas enters the straws and flows through them into the upstream plenum. From the upstream plenum the gas flows into the volume surrounding the straws, and from there the gas exhausts to the outside, bubbling through small jars of mineral oil.
The gas mixture used is 50$\%$ argon and 50$\%$ carbon dioxide at atmospheric pressure. 
This gas mixture was chosen since its drift time characteristics provide good position resolution~\cite{VanHaarlem:2010yq}.
A small admixture (approximately 1$\%$) of isopropanol is used to prevent loss of performance due to aging\cite{KADYK1991436,VAVRA20031}. 
Five thermocouples are located in each plenum and used to monitor the temperature of the gas.
The downstream plenum is 2.54~cm deep, with a sidewall of ROHACELL\footnote{www.rohacell.com} and a final outer wall of aluminized Mylar film, and the upstream plenum is 3.18~cm deep, with a polycarbonate sidewall and a polycarbonate disc outer wall. 

The readout cables pass through the polycarbonate disc and the upstream plenum to reach the anode wires. 
The cables are connected in groups of 20 to 24 to transition boards mounted onto the polycarbonate disc; the disc also supports the connectors for the high-voltage boards. 
Preamplifiers~\cite{hdnote2515} are mounted on the high-voltage boards. The aluminum endplate, outer cylindrical wall of the chamber, aluminum components connecting the straws to the aluminum endplate and the inside walls of the straws are all connected to a common electrical ground. 
The anode wires are held at +2.1~kV during normal operation. 


\subsection[Forward Drift Chamber]{Forward Drift Chamber
\label{sec:fdc} }

The Forward Drift Chamber (FDC) consists of 24 disc-shaped planar drift chambers of 1~m diameter \cite{FDC_NIM}.
They are grouped into four packages inside the bore of the spectrometer magnet.
Forward tracking requires good multi-track separation due to the
high particle density in the forward region.
This is achieved via additional cathode strips on both sides of the wire plane allowing for a  reconstruction of a space point on the track from each chamber. 
The FDC registers particles emitted into polar angles as low as $1^\circ$ and up to $10^\circ $
with all the chambers, while having partial coverage up to $20^\circ$.

One FDC chamber consists of a wire plane with cathode planes on either sides at a distance of $5$~mm from the wires (Fig.~\ref{FDC_OneCell}).
\begin{figure}[tbp]
\begin{center}
%\includegraphics[width=0.75\textwidth]{figures/FDC_OneCell.jpg}  
\includegraphics[width=0.95\textwidth]{figures/FDC_OneCell2.png} 
\caption{\label{FDC_OneCell}
Artist rendering of one FDC chamber showing components. From top to bottom: upstream cathode, wire frame, downstream cathode, ground plane that separates the chambers. The diameter of the active area is $1$~m.
}
\end{center}
\end{figure}
The frame that holds the wires is made out of ROHACELL with a thin
G10 fiberglass skin in order to minimize the material and
allow low energy photons to be detected in the outer electromagnetic calorimeters.

The wire plane has sense ($20~\mu$m diameter) and field ($80$~$\mu$m) wires $5$~mm apart, forming a field cell of $10\times 10$~mm$^2$. 
To reduce the effects of the magnetic field, 
a ``slow" gas mixture of $40\%$~Ar and $60\%$~CO$_2$ is used.
A positive high voltage of about $2.2$~kV is applied to the sense wires and a negative high voltage of $0.5$~kV to the field wires. 
The cathodes are made out of $2$-$\mu$m-thin copper strips on Kapton foil with a pitch of $5$~mm, and are held at ground potential. The strips on the two cathodes are arranged at $30^\circ $ relative to each other and at angles of $75^\circ $ and $105^\circ $ angle with respect to the wires.

The six chambers of a package are separated by thin aluminized Mylar.
Each chamber is rotated relative to the previous one by $60^\circ $.
The total material of a package in the sensitive area corresponds to $0.43\%$ radiation lengths, with about half of that in the area along the beam line that has no copper on the cathodes.
The sense wires in the inner area of $6-7.8$~cm diameter (depending on the distance of the package to the target) are increased in thickness from $20$~$\mu$m to $\sim 80$~$\mu$m, which makes them insensitive to the high rates along the beam.
The distance between the first and last package is $1.69$~m. 
All chambers are supplied with gas in parallel. 
In total, $2,304$ wires and $10,368$ strips are read using charge preamplifiers with $10$~ns peaking time, with a gain of $0.77$~mV/fC for the wires and $2.6$~mV/fC for the strips.

\subsection{Electronics \label{sec:dcelectronics}}
The high voltage (HV) supply units used are CAEN A1550P\footnote{www.caen.it}, with noise-reducing filter modules added to each crate chassis. 
The low voltage (LV) supplies are Wiener MPOD MPV8008\footnote{www.wiener-d.com}. 
The preamplifiers are a custom JLab design based on an ASIC~\cite{hdnote2515}
with 24 channels per board; the preamplifiers are charge-sensitive, capacitively coupled to the wires in the CDC and FDC, and directly coupled to strips in the FDC. 

Pulse information from the CDC anode wires and FDC cathode strips are obtained and read out using 72-channel 125 MHz flash ADCs (FADCs) \cite{Visser2008,5873864}. These use Xilinx\footnote{www.xilinx.com} Spartan-6 FPGAs (XC6SLX25) for signal digitization and data processing with 12 bit resolution.
Each FADC receives signals from three preamplifiers. 
The signal cables from different regions of the drift chambers are distributed between the FADCs in order to share out the processing load as evenly as possible.  

The FADC firmware is activated by a signal from the \gx{} trigger. The firmware then computes the following quantities for pulses observed above a given threshold within a given time window: pulse number, arrival time, pulse height, pulse integral, pedestal level preceding the pulse, and a quality factor indicating the accuracy of the computed arrival time. 
Signal filtering and interpolation are used to obtain the arrival time to the nearest 0.8~ns. 
The firmware performs these calculations both for the CDC and FDC alike, and uses different readout modes to provide the data with the precision required by the separate detectors. 
For example, the CDC electronics read out only one pulse but require both pulse height and integral, while the FDC electronics read out up to four pulses and do not require a pulse integral.  

The FDC anode wires are read out using the JLab pipeline F1 TDC\cite{hdnote1021} with a nominal least count of 120~ps. 

\subsection[Gas system]{Gas system \label{sec:gas}}
Both the CDC and FDC operate with the same gases, argon and CO$_{2}$. Since the relative mixture of
the two gases is slightly different for the two tracking chambers, the gas system has two separate but identical mixing stations. There is one gas supply of argon and CO$_{2}$ for both mixing stations. A limiting opening in the supply
lines provides over-pressure protection to the gas system, and filters in the gas lines provide protection against potential
pollution of the gas from the supply. Both gases are mixed using mass flow controllers (MFCs) that can be 
configured
to provide the desired mixing ratio of argon and CO$_{2}$.  MFCs and control electronics from
BROOKS Instruments\footnote{BROOKS Instruments, https://www.brooksinstrument.com/en/products/mass-flow-controllers.} are used throughout.

The mixed gas is filled into storage tanks, with one tank for the CDC and another for the FDC. The pressures are
regulated by controlling the operation of the MFCs with a logic circuit based on an Allen-Bradley ControlLogix system\footnote{Allen-Bradley, https://ab.rockwellautomation.com/}
 that keeps
the pressure in the tank between 10 and 12~psi. The tank serves both as a reservoir and a buffer.
A safety relief valve on each tank
provides additional protection against over-pressure. While the input pressure to the MFC is at 40~psi, the pressure after
the MFC is designed to always be less than 14~psi above atmospheric pressure. After the mixing tank, a provision is
built into the system to allow the gas to pass through an alcohol bath to add a small amount of alcohol gas to the gas mixture.
This small admixture of alcohol protects the wire chambers from aging effects caused by radiation exposure from the beam.
This part of the gas system is located above ground in a separate gas shed, before the gas mixture is transported
to the experimental hall via polyethylene pipes.

Additional MFCs in the hall allow the exact amount of gas provided to the chambers to be specified: one MFC for the CDC and another 
four MFCs for the individual FDC packages. The CDC is operated with a flow of 1.0~l/m, while each FDC package is operated with
a flow of 0.1~l/m. To protect the chambers from over-pressure, there is a bypass line at the input to the detectors that
is open to the atmosphere following a bubbler containing mineral oil. The height of the oil level determines the maximum possible gas pressure at
the input to the chambers. There is a second bubbler at the output to protect against possible air back-flow into
the chamber. The height of the oil above the exhaust line determines the operating pressure inside the chambers.

Valves are mounted at many locations in the gas system to monitor various pressures with a single pressure sensor. The pressures of all six FDC chambers are monitored, as well as the CDC gas at the input, downstream gas plenum and the exhaust. 
A valve in the exhaust line can be used to divert some gas from the chamber to an oxygen sensor. Trace quantities of oxygen will reduce the gas gain and reduce tracking efficiency. The oxygen levels in the chamber are below 100~ppm.

\subsection{Calibration, performance and monitoring \label{sec:dccalib}}
Time calibrations for the drift chambers are used to remove the time offset due to the electronics, so that after calibration the earliest possible arrival time of the pulse signals is at 0~ns. These offsets and the function parameters used to describe the relationship between the pulse arrival time and the closest distance between the track and the anode wire are obtained for each session of data taking. 

The CDC measures the energy loss, $dE/dx$, of tracks over a wide range of polar angles, including recoiling target protons as well as more forward-going tracks. Gain calibrations are made to ensure that $dE/dx$ is consistent between tracking paths through different straws and stable over time. 
The procedure entails matching the position of the minimum ionizing peak for each of the 3522 straws, and then matching the $dE/dx$ at 1.5~GeV/c to the calculated value of 2.0~ keV/cm. This takes place during the early stages of data analysis. Gain calibration for the individual wires is performed each time the HV is switched on and whenever any electronics modules are replaced. Gain calibration for the chamber as a whole is performed for each session of data taking; these sessions are limited to two hours as the gain is very sensitive to the atmospheric pressure. Position calibrations were necessary to describe the small deflection of the straw tubes midway along their length; these were performed in 2016 and repeated in 2017, with no significant difference found between the two sets of results.  Position resolution from the CDC is of the order of 130~$\mu$m and its detection efficiency per straw is over 98\% for tracks up to 4~mm from the CDC wire. The efficiency decreases as the distance between the track and the wire increases, but the close-packing arrangement of the straw tubes and the large number of straws traversed by each track compensate for this. 

For the FDC system, an internal per-chamber calibration process is first performed to optimize the track position accuracy.  
In the FDC the avalanche created around the wire is seen in three projections: on the two cathodes and on the wires.
The drift time information from the wires is used to reconstruct the hit position perpendicular to the wire.
The strip charges from the two cathodes are used to reconstruct the avalanche position along the wire. 
The same strip information can be used to reconstruct the avalanche position perpendicular to the wire,  which, due to the proximity of the avalanche to the wire, is practically the wire position, as illustrated in Fig~\ref{FDC_wires_from_strips}.
\begin{figure}[tbp]
\begin{center}
\includegraphics[width=0.95\textwidth]{figures/FDC_wires_from_strips1.pdf}  
\caption{\label{FDC_wires_from_strips} Wire (avalanche) positions reconstructed from the strip information on the two cathodes in one FDC chamber. Only one quarter of the chamber is shown in this figure.
}   
\end{center}  
\end{figure}
This strip information is used to align the strips on the two cathodes with respect to the wires. 
At the same time, the residuals of the reconstructed wire positions are an estimate of the strip resolution.
The resolutions of the detector were reported earlier \cite{FDC_NIM}. 
The strip resolution along the wires, estimated from the wire position reconstruction, varies between $180$ and $80$~$\mu$m, depending on the total charge induced on the strips. The drift distance is reconstructed from the drift time with a resolution between $240$ and $140$~$\mu$m
depending on the distance of the hit to the wire in the $0.5-4.5$~mm range.  

Position offsets and package rotations were determined for both drift chamber systems, first independently, and then together, using the alignment software MILLEPEDE\cite{millepede} in a process described in \cite{GlueXCDCNIM} and in \cite{MikeStaib_thesis}.

Online monitoring software enables shift-takers to check that the number of channels recording data, the distribution of signal arrival times, and the  $dE/dx$ distribution are as expected. 


%\subsection{Summary \label{sec:dcsummary}}
 
 

% \clearpage    % avoid formatting problems with empty sections
\section[Performance of the charged-particle-tracking system]{Performance of the charged-particle-tracking system \label{sec:trackingperformance}}
\subsection{Track reconstruction}

The first stage in track reconstruction is pattern recognition.  Hits in adjacent
 layers in the FDC in each package are formed into track segments that are 
linked together with other segments in other packages to form FDC track 
candidates using a helical model for the track parameters.
Hits in adjacent rings in the axial layers of the CDC are also associated into 
segments that are linked together with other segments in other axial layers
and fitted with circles in the projection perpendicular to the beam line. Intersections between these circles and the stereo wires are found and a linear fit is performed to find a $z-$position near the beamline and the tangent to the dip
 angle $\lambda=\pi/2-\theta$.  These parameters, in addition to the circle fit 
parameters, form a CDC track candidate for each set of linked axial and stereo 
layers. Candidates that emerge from the target, and pass through both FDC and CDC in the  $5^\circ-20^\circ$ range, are linked together.

The second stage uses a Kalman filter \cite{KalmanFilter, KalmanFilter2} to find the fitted track parameters
\{z,D,$\phi$,$\tan\lambda$,$q/p_T$\}
at the position of closest approach of the track to the beam line. The track candidate parameters are used as an initial guess, where D is the signed distance of closest approach to the beam line.  The Kalman filter proceeds in steps from the hits farthest from the beam line toward the beam line. Energy loss and multiple scattering are taken into account at each step along the way, according to a map of the magnetic field within the bore of the solenoid magnet.
For the initial pass of the filter, the drift time information from the 
wires is not used.  Each particle is assumed to be a pion, except for low momentum track 
candidates ($p<0.8$~GeV/$c$), for which the fits are performed with a proton hypothesis.

The third stage matches each fitted track from the second stage to either
the Start Counter, the Time-of-Flight scintillators, the Barrel Calorimeter, or
the Forward Calorimeter to determine a start time t0 so that the drift time to
each wire associated with the track could be used in the fit. Each track is refitted with
the drift information, separately for each value of mass for particles in the set \{$e^\pm,\pi^\pm,K^\pm,p^\pm$\}.

\subsection{Momentum and vertex resolution}

The momentum resolution as a function of angle and magnitude for pions and 
protons is shown in Fig.~\ref{fig:dp_p}.  The angular resolution is shown in 
Fig.~\ref{fig:angle res}.


\begin{figure}[tbp]
\begin{center}
\includegraphics[width=0.45\textwidth]{figures/PionMomentumResolution.pdf}
\includegraphics[width=0.45\textwidth]{figures/ProtonMomentumResolution.pdf}
\caption{\label{fig:dp_p} (Left) Momentum resolution for $\pi^-$ tracks.
(Right) Momentum resolution for proton tracks.}
\end{center}
\end{figure}

\begin{figure}[tbp]
\begin{center}
\includegraphics[width=0.45\textwidth]{figures/PionThetaResolution.pdf}
\includegraphics[width=0.45\textwidth]{figures/PionPhiResolution.pdf}
\caption{\label{fig:angle res} (Left) Polar angle resolution for $\pi^-$ tracks.
(Right) Azimuthal angle resolution for $\pi^-$ tracks.
The resolutions are plotted as a function of the polar angle, $\theta$.}
\end{center}
\end{figure}

The thin windows of the cryogenic target and the exit window of the target
vacuum chamber provide a means to estimate the 
vertex resolution of the tracking system.  Pairs of tracks from empty target measurements are used to reconstruct these windows as illustrated in 
Fig.~\ref{fig:z-vertex}. The distance of closest approach between two tracks, $d$, was required to be less than 1~cm. The vertex position 
is at the mid-point of the line segment (of length $d$) defined by the points of closest approach for each track.
The estimated $z$-position resolution is 3~mm.

\begin{figure}[tbp]
\begin{center}
\includegraphics[width=0.7\textwidth]{figures/ZVertex.pdf}  
\caption{\label{fig:z-vertex} Reconstructed vertex positions within 1 cm radial
 distance with respect to the beam line for an empty target measurement.  The curve shows the result of a fit to the vertex distribution used to determine the vertex
resolution. 
}   
\end{center}  
\end{figure}


\input{calorimeters}
% \clearpage    % avoid formatting problems with empty sections
\input{scintillators}
%=======================+=========================
%==============  Trigger    ================
%=================================================\

\section[Trigger]{Trigger \label{sec:trig}}
The goal of the \gx{} trigger is to accept most high-energy hadronic interactions while reducing the background rate induced by electromagnetic and low-energy hadronic interactions to the level acceptable 
by the data acquisition system (DAQ).  The main trigger algorithm is based on measurement of energy depositions in the FCAL and BCAL as described in Ref.~\cite{somov_l1,somov_l11}. Supplementary triggers can also use hits from scintillator detectors, such as the PS, tagging detectors, ST, TOF, and TAC.

\subsection{Architecture \label{sec:trigarchitecture}}
The \gx{} trigger system\cite{GlueX:2013twa} is implemented on special-purpose programmable pipelined electronics modules using Field-Programmable Gate Arrays (FPGAs) designed at Jefferson Lab.  The \gx{} trigger and readout electronics are hosted in VXS (ANSI/VITA 41.0) crates; VXS is an extension of the VME/VME64x architecture that implements high-speed backplane lines used to transmit trigger information. 

A layout of the trigger system is presented in Fig.~\ref{fig:trig}. Data from the FCAL and BCAL are sent to  FADC modules~\cite{Dong:2007}, situated in 12 and 8 VXS crates, respectively, and are digitized at the sampling rate of 250 MHz. The digitized amplitudes are used for the trigger and are also stored in the FADC FPGA-based pipeline for subsequent readout via VME.
Digitized amplitudes are summed for all 16 FADC250 channels in each 4 ns sampling interval and are transmitted to the crate trigger processor (CTP) module, which sums up amplitudes from all FADC boards in the crate. The sub-system processor (SSP) modules located in the global trigger crate receive amplitudes from all crates and compute the total energy deposited in the FCAL and BCAL. The global trigger processor (GTP) module collects data from the SSPs, performs computation of different trigger equations, and makes the trigger decision. The core of the trigger system is the trigger supervisor (TS) module, which receives the trigger information from the GTP and distributes triggers to the electronics modules in all readout 
crates in order to initiate the data readout. There are 55 VXS crates in total in \gx{} (26 with FADC250s, 14 with  FADC125s, 14 with F1 TDCs, and 1 CAEN TDC). The TS also provides a synchronization of all crates and provides a 250 MHz clock signal. The triggers and clock are distributed through the trigger distribution (TD) module in the trigger distribution crate and the trigger interface (TI) module and signal distribution (SD) module in each readout crate. The \gx{} trigger system provides a fixed latency. The longest trigger distribution time arises from the crates in the tagger hall, about 3.3 $\mu$s. 
The smallest rewritable readout buffer, where hits from the detector are stored, corresponds to about 3.7$\mu$s for the F1 TDC module. The trigger jitter does not exceed 4 ns.


\begin{figure}[tbp]
\begin{center}
\includegraphics[width=0.75\textwidth]{figures/125_Somov-f1.pdf}  
\caption{Schematic view of the Level-1 trigger system of the \gx{} experiment. Description of the electronics boards is given in the text.} \label{fig:trig}
\end{center}
\end{figure}

\subsection{Trigger Types \label{sec:triggers}}

The \gx{} experiment uses two main trigger types: the pair spectrometer trigger, and the physics trigger based on energy depositions in the BCAL and FCAL. The 
pair spectrometer trigger is used to measure the flux of beam photons. This trigger requires a time coincidence of hits in the 
two arms of the PS detector, described in Section~\ref{sec:ps}. The physics triggers are generated when the FCAL and BCAL energies  satisfy the following conditions: (1) $2\cdot E_{\rm FCAL} + E_{\rm BCAL} > 1\;{\rm GeV}$,  $E_{\rm FCAL} > 0\; {\rm GeV}$ and (2) $E_{\rm BCAL} > 1.2\;{\rm GeV}$. The latter trigger type is used to accept events with large transverse energy released in the BCAL, such as decays of $J/\psi$ mesons. 

Several other trigger types were implemented for efficiency studies and detector calibration. 
Efficiency of the main production trigger was studied using a trigger based on the coincidence of hits from the ST and TAGH, detectors not used in the main production trigger. A combination of the PS and TAC triggers was used for the acceptance calibration of the PS, described in Section~\ref{sec:ps_flux}. Ancillary minimum-bias random trigger and calorimeter LED triggers were collected concurrently with data taking.

\subsection{Performance \label{sec:trigperformance}}
The rate of the main physics triggers as a function of the PS trigger rate is shown in Fig.~\ref{fig:trig_rate}.
The typical rate of the PS trigger in spring 2018 was about 3 kHz, which corresponds to a photon beam flux of $2.5\cdot 10^7\; \gamma/{\rm sec}$ in the \gx{} energy range of interest. The total trigger rate was about 40 kHz. The rates of the random trigger and each of the LED calorimeter triggers were set to 100 Hz and 10 Hz, respectively. The electronics and DAQ were running with a livetime close to 
$100 \%$, collecting data at a rate of 600 MB per second.
The trigger system can operate at significantly higher rates, which is considered for the next phase
of the GlueX experiment. The dead time of the trigger and DAQ systems at the trigger rate of 80 kHz
was measured to be about $10 \%$. The largest contribution to the dead time comes from the hit processing
time of readout electronics modules. 

%In spring 2018, data were collected at the luminosity %corresponding to the photon beam flux of about $2.5\cdot 10^7\; %\gamma/{\rm sec}$ in the GlueX energy range of interest. The %typical rate of the PS trigger 

\begin{figure}[tbp]
\begin{center}
\includegraphics[width=0.75\textwidth]{figures/Rate_diamond_2018.png}  
\caption{Rate of the main production triggers as a function of the PS rate: FCAL and BCAL trigger (boxes), BCAL trigger (triangles), the total trigger rate (circles). The vertical arrow indicates the run condition corresponding to the spring run of 2018.} \label{fig:trig_rate}
\end{center}
\end{figure}

%=======================+=========================
%==============  DAQ   ================
%=================================================\

\section[Data Acquisition]{Data acquisition \label{sec:daq}}

%\subsection{Architecture \label{sec:daqarchitecture}}

%\subsection{Performance \label{sec:daqperformance}}

The GlueX data acquisition uses the CEBAF Data Acquisition (CODA) framework. CODA is a software toolkit of applications and libraries that allows one to build customized data acquisition systems based on distributed commercial networks. A detailed description of CODA software and hardware can be found in Ref.\,\cite{CLAS12DAQ}. 
The maximum readout capability of the electronics in the VME/VXS crate is 200 MB/s per crate and the number of crates producing data is about 55.
The data from the electronic modules are read via the VME back-plane (2eSST, parallel bus) by the crate readout controller (ROC), which is a single board computer running Linux.
The \gx~ network layout and data flow are shown on Fig.~\ref{fig:CODA}.
Typical data rates from a single ROC are in the range of 20--70~MB/s, depending on the detector type and trigger rate.
The ROC transfers data over 1~Gbit Ethernet links to Data Concentrators (DC) using buffers containing event fragments from 40 triggers at a time. Data Concentrators are programs, which build partial events received from 10-12 crates and run on a dedicated computer node.
The DC output traffic of 200-600 MB/s is routed to the Event Builder (EB) to build complete events.
The Event Recorder (ER), which is typically running on the same node as an Event Builder, writes data to local data storage.
To date, GlueX has collected data at a rate of 500--900 MB/s, which allows the ER to write out to a single output stream. The system is expandable to handle higher luminosity with rates of 1.5--2.5~GB/s. In this case, the ER must write multi-stream data to several files in parallel.
All DAQ computer nodes are connected to both a 40 Gb Ethernet switch and a 56 Gb Infiniband switch.
The Ethernet network is used exclusively for DAQ purposes: receiving data from detectors, building events, and writing data to disk, 
while the Infiniband network is used to transfer events for online data quality monitoring. 
This allows decoupling DAQ and monitoring network traffic.
The live time of the DAQ is in the range of 92--100\%. The dead time comes from readout electronics and depends on the trigger rate.  
The software part of DAQ has no dead time during the run, but it appears while stopping and starting the run, which takes about between 2-8 minutes. 



\begin{figure}[tbp]
\begin{center}
\includegraphics[width=0.75\textwidth]{figures/DAQ_coda.pdf}  
\caption{ \label{fig:CODA}
Schematic DAQ configuration for \gx. The high-speed DAQ connections between the ROCs and the ER are contained within an isolated network. The logical data paths are indicated by arrows,
although physically they are routed through the 40 Gbit ethernet switch.  The Online monitoring system uses its own separate 56 Infiniband switch.}

\end{center}
\end{figure}

%=======================+=========================
%================  Controls  ================
%=================================================

\section[Slow controls]{Slow controls \label{sec:controls}}
\GX{} must monitor 
and control tens of thousands of different variables that define the state of the experimental hardware. Different variable values need to be acquired, displayed, archived, and used as inputs to control loops continually with a high degree of reliability. For \gx, approximately 90,000 variables are archived, and many more are monitored.

\subsection{Architecture \label{sec:controlsarchitechture}}
The \gx{} slow control system consists of three layers. The first layer consists of the remote units such as high voltage or low voltage power chassis, magnet power supplies, temperature controller, LabView applications, and PLC-based applications, which directly interact with the hardware and contain almost all the control loops. The second layer is the Supervisory Control and Data Acquisition (SCADA) layer, which is implemented via approximately 140 EPICS Input/Output Controllers (IOC's). This layer provides the interface between low level applications and higher level applications via the EPICS ChannelAccess protocol. The highest level, referred as the Experiment Control System (ECS), contains applications such as Human-Machine Interfaces, the alarm system, and data archiving system. This structure allows for relatively simple and seamless addition and integration of new components into the overall controls system.    

\subsection{Remote Units \label{sec:controlsinterface}}
\gx{} uses a variety of commercial units to provide control over the hardware used in the experiment. For instance, most detector high voltages are provided by the CAEN SYx527 voltage mainframe,\footnote{https://www.caen.it/subfamilies/mainframes/} while the low and bias voltages are provided by boards residing in a Wiener MPOD chassis\footnote{http://www.wiener-d.com/sc/power-supplies/mpod--lvhv/mpod-crate.html}. These two power supply types provide most voltages for detector elements with the exception of Tagger Microscope and Forward Calorimeter where custom systems were developed that provide voltage regulation and interact with the EPICS-based layer through higher level interfaces using custom protocols. See Sections.~\ref{sec:TAGM} and \ref{sec:fcal} for more details.  

Various beam line devices need to be moved during beam operations. Stepper motors are used to move motorized stages via Newport XPS universal multi-axis motion controllers\footnote{https://www.newport.com/c/xps-universal-multi-axis-motion-controller.} that allow for execution of complex trajectories involving multiple axes. All stage referencing, motion profile computations, and encoder-based closed-loop control occur within the controller chassis after the basic parameters, such as positions and velocities, are provided by the user via a TCP/IP-based interface to EPICS.   

While installing complex systems, such as a superconducting magnet that requires large numbers of input and output channels and sophisticated logic, custom controls systems were developed for each particular system. For those cases, we used Allen-Bradley CompactLogix and ControlLogix PLC systems\footnote{https://ab.rockwellautomation.com.}. These systems are designed for industrial operations, allow modular design, provide high reliability, and require minimal maintenance. All controls loops are programmed within the PLC application, and are interfaced with EPICS through a TCP/IP-EtherNet/IP-proprietary protocol to allow access by higher level applications to process variables served by the PLC's.  

The cryogenic target and the superconducting solenoid employ National Instruments LabView applications. The target controls use both custom-made and vendor-supplied hardware that include built-in remotely-accessible control systems and an NI CompactRIO\footnote{https://www.ni.com/en-us/shop/compactrio.html} chassis. This chassis communicates with the hardware and serves variables using an internal ChannelAccess server and an EPICS IOC running on the CompactRIO controller, as described in Sec.~\ref{sec:target}. A National Instruments PXI high-performance system\footnote{https://www.ni.com/en-us/shop/pxi.html} is used to collect data from different sensors as described in Sec.~\ref{sec:solenoid}. 

\begin{landscape}
 \begin{figure}[tbp]
\begin{center}
\includegraphics[height=10cm,bb=35 480 535 760,clip=true]{figures/GlueX_CSS_overview.pdf}
%\includegraphics[height=8cm,clip=true]{figures/GlueX_CSS_overview.pdf}
\caption{Top-level graphical interface for the beamline. This screen provides information on beam currents and rates, radiators, magnet status, target condition, background levels, etc.
\label{fig:GlueX_CSS_overview}
}
\end{center}
\end{figure}
\end{landscape}

\subsection{Supervisory Control and Data Acquisition layer \label{sec:archiver}}
The SCADA layer is the middle layer that distributes the process variables allowing the higher level --and sometimes lower level-- applications to use various process variables of the Hall-D control system. This layer is based on EPICS and uses the ChannelAccess protocol to publish the values of the variables over Ethernet. Because the accelerator controls also use EPICS, efficiently exchange the information between the experiment and accelerator operations is achieved. Several dozen software IOC processes running on hosts in the experiment control room collect data from different components of the lowest layer. Each IOCs is configured to communicate using the protocol appropriate for the remote units with which data exchange is needed. For instance, the IOC controlling the voltage for the FDC detector needs to be able to communicate with the Wiener MPOD and CAEN SYx527 voltage chassis. Although the middle layer is primarily used to distribute data between different applications, that layer also contains some EPICS-based applications running on IOC's that provide different control loops and software interlocks.  For instance, the low-voltage power supplies for the FDC detector (see Sec. \ref{sec:fdc}) are shut off if the temperature or the flow of the coolant in the chiller falls outside of required limits. 
\subsection{Experiment Control System \label{sec:alarms}}
The highest level of controls contains applications that archive data, display data in interactive GUIs and as stripcharts, alarm and notify shift personnel and experts in case problems occur, and interface with the CODA-based data acquisition system (Sec.~\ref{sec:daq}).
An example of such a GUI is the beamline overview screen, shown in Fig.\,\ref{fig:GlueX_CSS_overview}. Many of the buttons of the GUI are active and allow access to other GUIs.
Display management and the alarm system for \gx{} controls are based on Controls System Studio (CSS),\footnote{http://controlsystemstudio.org/}  which is an Eclipse-based toolkit for operating large systems. CSS is well suited for systems that use EPICS as an integral component. Although CSS provides an archiving engine and stripcharting tools, the MYA archiver~\cite{Slominski:2009icaleps} provided by the JLab accelerator software group was employed with its tools for displaying the archived data as a time-series. Display management for \gx{} controls is within the CSS BOY~\cite{Chen:2011icaleps} environment, which allows system experts to build a sophisticated control screens using standard widgets. The alarm system is based on the CSS BEAST\cite{Kasemir:2009icaleps} alarm handler software, which alerts shift personnel of problems with the detector, and notifies a system expert if the problems are not resolved by shift personnel.   
%=======================+=========================
%================  Online  ================
%=================================================

%------------------------------------------------------------------
\section[Online computing system]{Online computing system \label{sec:online}}

This section describes the \GX ~software and computing systems that are used for both data monitoring and transport to the tape system for permanent storage.

%------------------------------------------------------------------
\subsection{Monitoring \label{sec:onlinemonitoring}}

The Online Monitoring system (OMS) consists of multiple stages that provide for immediate monitoring of the data, as well as near-term monitoring ($\sim$hours after acquisition). The immediate monitoring is based on the \textit{RootSpy} system\cite{rootspy} written for use in \GX. Figure \ref{fig:online_monitoring_processes} shows a diagram of the processes involved in the RootSpy system and how they are coupled to the DAQ system. The Event Transfer (ET) is part of the CODA DAQ system\cite{coda} and is used to extract a copy of a portion of the data stream without interfering with the DAQ itself. The monitoring system itself utilizes a secondary ET in order to minimize connections to the RAID server that is running the Event Recorder process.

\begin{figure}[tbp]
\begin{center}
\includegraphics[width=0.99\textwidth, clip,trim=1.5cm 0.9cm 1.7cm 0.8cm]{figures/online_monitoring_processes.pdf}
\caption{\label{fig:online_monitoring_processes}Diagram of the various processes distributed across several computers in the counting house that make up the online monitoring system. DC, EB, and ER represent the Data Concentrator, Event Builder, and Event Recorder processes respectively in the CODA DAQ system.}   
\end{center}  
\end{figure}

RootSpy was developed at JLab for \GX, though its design is not \GX ~specific. The system is run on a small farm of computers\footnote{The online monitoring farm consists of eight 2012 era Intel x86\_64 computers with 16 cores+16ht plus six 2016 era Intel x86\_64 computers with 36 cores + 36ht. The monitoring farm uses 40 Gbps (QDR) and 56 Gbps(FDR) IB for the primary interconnect. Note that the DAQ system uses a separate 40 Gbps ethernet network that is independent of the farm.} in the counting house, each processing a small part of the data stream. In total, about 10\% of the data is processed for the low level occupancy plots while roughly 2\% is fully reconstructed for higher level plots. The CODA ET software system is used to distribute the data among the farm computers. Each farm node generates histograms which RootSpy gathers and combines before displaying them to shift workers in a GUI.
%Figure \ref{fig:online_monitoring_rootspy} shows a screen capture of the main RootSpy GUI window along with a window displaying the reference plot for the currently displayed image.
Plots are displayed via a set of ROOT macros, each responsible for drawing a single page. Most macros divide the page into multiple sections so that multiple plots can be displayed on a single page. Figure \ref{fig:online_monitoring_PID} shows an example of a high level monitoring plot where four invariant mass distributions are shown with fits. Values extracted from the fits are printed on the plots for easy quantitative comparison to the reference plot. 

Shift workers are presented with a live plot alongside a reference plot by which to compare. Shift workers may assign a live plot as the new reference using a button on the RootSpy GUI. Relevant experts for the given plot are notified via e-mail when a new reference is made thus, providing for expert oversight of the reference plots.

ROOT macros are passed into the RootSpy GUI using the same xMsg\cite{xmsg} publish subscribe messaging system used to transport histogram objects. The macros are compiled into plugins as C++ strings by the build system. The build system recognizes ROOT macro files in the plugin source directory (via the .C suffix) and automatically generates C++ code that contains the macro and code to register it with the RootSpy system. This is done to ensure that a macro is always in sync with the histograms it displays since the former is linked in the same binary as the routines that produce the latter.

\begin{figure}[tbp]
\begin{center}
\includegraphics[width=0.99\textwidth, clip,trim=0.6cm 0.0cm 1.1cm 0.0cm]{figures/online_monitoring_PID.pdf}
\caption{\label{fig:online_monitoring_PID}Invariant mass distributions showing $\pi^\circ$, $\omega$, $\rho$, and $\phi$ particles. These plots were generated online in about 1hr 40min by looking at roughly 2\% of the data stream.}   
\end{center}  
\end{figure}

In addition to the RootSpy GUI, several other client programs exist that consume the histograms being produced by the RootSpy system. One of these is the RSTimeSeries program. This program periodically runs a subset of macros that contain special calls to insert data into an InfluxDB time series database. This provides a web-accessible strip chart of detector hit rates and reconstructed quantities (e.g. number of $\rho$'s per 1k triggers). The RSArchiver program gathers summed histograms and periodically writes them to a ROOT file for permanent archival. The archive file is later used to generate a set of image files that are displayed in the Plot Browser\footnote{https://halldweb.jlab.org/data\_monitoring/Plot\_Browser.html.} website. Plot Browser allows for easy comparison of plots for different runs as well as with similar plots produced during offline analysis. The first five files (100GB) of each run are automatically pinned to disk in the JLab Computing Center after being transported there for permanent tape storage. Jobs are automatically submitted for these files to the JLab SciComp farm to perform full reconstruction. The results of this are displayed in Plot Browser under the title ``Incoming Data.''


%------------------------------------------------------------------
\subsection{Data Transport and Storage \label{sec:onlineprocessing}}

\GX ~Phase I generated production data at rates up to 650MB/s. The data was temporarily stored on large RAID-6 disk arrays and then copied to an LT0 tape system for long term storage. Two RAID servers, each with four partitions were used in the Hall-D counting house. The partition being written to was rotated between runs through the use of symlinks. This helped to minimize head thrashing on disks by only reading from the partitions not currently being written to. Partitions were kept at approximately 80\% full and older files deleted only as needed to maintain this. This was to allow the monitoring farm easy access to the files for times when the beam was down. An additional copy of the first 3 files ($\sim1.5\%$) of each run was made to a smaller RAID disk so that an easily accessible sample of each run could be maintained in the counting house.

The data volumes stored to tape are shown in table \ref{tab:online_data_volumes} in units of petabytes(PB). Lines marked ``actual'' are values taken from the tape storage system. Lines marked ``model'' come from the \GX ~computing model\cite{gx3821}.

\begin{table}[]
    \centering
    \begin{tabular}{|l|c|c|c|c|c|}
    \hline
                           & \textbf{2016}  & \textbf{2017}  & \textbf{2018} \\
    \hline
    actual (raw data only) & 0.624 & 0.914 & 3.107 \\
    \hline
     model (raw data only) &       & 0.863 & 3.172 \\
    \hline
    \hline
    actual (production)    & 0.55  & 1.256 & 1.206 \\
    \hline
     model (production)    &       & 0.607 & 3.084 \\
    \hline
    \end{tabular}
    \caption{\GX ~Data volumes by year. All values are in petabytes(PB). Most years include two run periods. The lines marked \textit{model} are calculated from the \GX ~Computing Model\cite{gx3821}. ``Raw data only'' represents data generated by the DAQ system (not including the backup copy). ``Production'' represents all derived data including reconstructed values and ROOT trees. }
    \label{tab:online_data_volumes}
\end{table}


% \clearpage   % avoid formatting problems with empty sections
%=======================+=========================
%================  Reconstruction  ================
%=================================================


\section[Event reconstruction]{Event reconstruction \label{sec:reconstruction}} 

% Copy from GlueX-doc-3108
% "Production and Analysis of GlueX Data"
% TODO: UPDATE for 2017

\GX~uses the computer center batch farm at JLab to perform data monitoring, event reconstruction, and physics analyses.  For data monitoring, detector hit occupancies, calibration and reconstruction quality, and experimental yields and resolutions, are analyzed for several physics channels.  A subset of the data is monitored automatically as it is saved to tape.  Every few weeks, monitoring processes are launched on a subset of the data to study improvements from ongoing calibrations and reconstruction software improvements.  The histograms produced by these monitoring jobs are displayed on a website and ROOT files are available for download, enabling the collaborators to easily study the quality of the data. 

Every few months, a major reconstruction launch over all of the data is performed, linking hits in the various detector systems to reconstruct particles in physics events.  Monitoring plots from these launches are also published to the web. Finally, regular analysis launches over the reconstructed data are performed for the reactions requested by users on a web form. The results of these launches are saved in reaction-specific ROOT TTrees for further analysis.

For all launches, the reconstruction is run in a multi-threaded mode to make efficient use of the available computing resources. Fig.\,\ref{fig:offline_monitorA} shows the multithreaded scaling from our monitoring launches. The program performs near the theoretical limit for jobs that use a number of threads that is less than or equal to the number of physical cores on the processor. By using hyperthreads, a smaller but still significant gain is achieved.
%SWIF is used to manage the batch farm jobs, and is queried to study the performance of the launches.  Figure~\ref{fig:farm-time} shows how many batch farm jobs were in each processing state as function of time our latest reconstruction launch.
All file outputs are written to a write-through cache system, which is ultimately backed up to tape.

\begin{figure}[h!]\centering
%\includegraphics[width=0.7\textwidth]{figures/OfflineMonitor_PlotA.pdf}
\includegraphics[width=1.0\textwidth]{figures/rate_vs_nthread.pdf}
\caption[]{\label{fig:offline_monitorA}The scaling of program performance as a function of the number of processing threads. The computer used for this test consisted of 24 full cores (Intel x86\_64) plus 24 hyperthreads. The orange squares are from running multiple processes, each with 12 threads.} 
\end{figure}

\GX~ Phase I has recorded 1400 separate physics-quality runs, with a total data footprint of about 3 petabytes. Data were saved in 19-GB files, with all runs consisting of multiple files (typically $100$ or more per run). Fig.\,\ref{fig:production_overview} shows an overview of the different production steps for \GX~data, which are described in more detail in the following subsections.

\begin{figure}[hbt]\centering
%\includegraphics[width=0.8\textwidth]{figures/Production_generic.pdf}
\includegraphics[width=0.9\textwidth]{figures/production_overview_calib_v2.pdf}
\caption[]{\label{fig:production_overview}Production flowchart for \GX~data, illustrating analysis steps.} 
\end{figure}

%--------------------------------------------------------------------------
\subsection{Calibration \label{sec:reccalibration}}
During the acquisition of data, a unique run number is assigned to a period of data corresponding to less than about 2 hours of clock time, which may result in writing a couple hundred files. It is assumed that the detector changes very little during this period and therefore there will be no changes in the calibration constants.
Two types of calibration procedures are used, depending on the complexity of the calibration procedures.  Simple, well-understood calibrations such as timing alignment between individual channels and subdetectors or drift chamber gain and time-to-distance calibrations, can be performed with one file of data per run.  These procedures are executed either in the online environment or on the batch farm, and can be repeated as needed following any improvements in reconstruction algorithms or other calibrations.

More complicated calibration procedures, such as calorimeter gain calibration, require more data and are often iterative procedures, requiring several passes through the data.  The raw data are processed upon arrival on the batch farm, resulting in histograms or in selected event data files in EVIO \cite{EVIO} or ROOT-tree format.
Many of these outputs require that charged particle tracks are reconstructed. However, the computationally intensive nature of track reconstruction makes it a challenge to fully reconstruct all raw data immediately. Therefore, the full suite of calibration procedures is only applied to 10 - 20\% of the data.
Processing of the remaining data is mostly focused on separating out, or ``skimming,'' events collected by calibration triggers. 


%--------------------------------------------------------------------------
\subsection{Monitoring \label{sec:recmonitoring}}

In Fig.~\ref{fig:production_overview} the ``FULL  RAW DATA'' box represents experimental data that have been backed up to tape. The box labeled ``subset" represents the first five files of each run, which are run through offline monitoring processes. These monitoring jobs are first processed during the run to check the quality of the data, but are also processed after major changes to calibrations or software to validate those changes.
The resulting Reconstructed Events Storage (REST) files and ROOT histogram files are used for checking the detector and reconstruction performance.

%--------------------------------------------------------------------------
\subsection{Reconstruction \label{sec:recreconstruction}}

When the data have been sufficiently well calibrated, a full (production) pass of the reconstructed software on the physics quality data is performed. In the current total \GX~data set, about 1400 runs were deemed ``physics quality." The remaining runs were short runs related to engineering and commissioning tests of the experiment. The 1400 physics quality runs include the majority of the data recorded during the running period, representing about 3 petabytes. All these files were reconstructed using computing resources at several sites, equivalent to more than 20 million core-hours combined. This produced more than $500$ terabytes of REST data files. The large reduction in size from collected event data to physics data files (about a factor of six) permits faster and more efficient physics analyses of the data.
%, and is also small enough to be fully exported to off-site computer centers, see section~\ref{sec:remote-dist} (not really).

During the REST production, a series of detector studies were performed that required access to raw data and that would not be possible on the reconstructed data alone. Many improvements to software and detector calibration resulted from these studies. Similar studies can be made with simulated data to match and assess the detector acceptance.

%--------------------------------------------------------------------------
\subsection{Offsite reconstruction}
\label{sec:recoffsite}

Production processing of \GX~data uses offsite high-performance computing resources in addition to the onsite computing farm at JLab, specifically, the National Energy Research Supercomputing Center (NERSC) and the Pittsburgh Supercomputing Center (PSC). For NERSC, the total allocation used for the academic year 2018-2019 was 53M NERSC units, which was used to process 70.5k jobs. This is equivalent to approximately 9M core-hours on a Intel x86\_64 processor. The jobs were run on NERSC's Cori II system, which is comprised of KNL (Knight's Landing) processors. The PSC allocation was awarded through the XSEDE\footnote{https://www.xsede.org.} allocation system in the last quarter of calendar year 2019 for 5.9 MSU's. Only 0.85M SU's were used in 2019 to run 7k jobs on the PSC Bridges system or about 10\% of the number processed at NERSC. Figure~\ref{fig:production_offsite_rate_vs_nthreads_NERSC} shows how the event processing rates scaled with the number of processing threads for both NERSC and PSC. Jobs run at both of those sites were assigned entire nodes so the number of processing threads used was equal to the total number of hardware threads.

\begin{figure}[htb]\centering
\includegraphics[width=0.49\textwidth]{figures/production_offsite_rate_vs_nthreads_NERSC.pdf}
\includegraphics[width=0.49\textwidth]{figures/production_offsite_rate_vs_nthreads_PSC.pdf}
\caption[]{\label{fig:production_offsite_rate_vs_nthreads_NERSC}Event processing rate versus number of threads for reconstruction jobs on NERSC Cori II (left) and PSC Bridges (right). The slope changes in the NERSC plot is due to the KNL architecture, which had four hardware threads per core. For PSC Bridges, hyper-threading is disabled and the plot shows a single slope.} 
\end{figure}

Container and distributed file system technologies were used for offsite processing. The software binaries as well as calibration constants, field maps, etc. were distributed using the CERN-VM-file system (CVMFS). 
The binaries were all built at JLab using a CentOS7 system. A very lightweight Docker container was made based on CentOS7 that had only a minimal number of system RPMs\footnote{RedHat Package Management, https://access.redhat.com/documentation/en-us/red\_hat\_enterprise\_linux/5/html/deployment\_guide/ch-rpm} installed. All other software, including third-party packages such as ROOT, were distributed via CVMFS. This meant changes to the container itself were very rare (about once per year). The Docker container was pulled into NERSC's Shifter system without modification. The same container was used to create a Singularity container used at both PSC and on the Open Science Grid (OSG) for simulation jobs.


Raw data ware transferred from JLab to the remote sites using Globus\footnote{https://opensciencegrid.org/technology/policy/globus-toolkit.},  which uses GridFTP. The Globus tasks were submitted and managed by the SWIF2 workflow tool written by the JLab Scientific Computing group. SWIF2 was needed to manage the data retrieval from tape, for transfer to the remote site, for submission of remote jobs, and for transfer of processed data back to JLab. Disk space limitations at both JLab and the remote sites meant only a portion of the data set could be on disk at any one time. Thus, SWIF2 had to manage the jobs through all stages of data transfer and job submission.

%--------------------------------------------------------------------------
\subsection{Analysis \label{sec:recanalysis}}

The full set of reconstructed (REST) data is too large to be easily handled by individual analyzers. For that reason, a system was developed to analyze data at JLab and extract reaction-specific ROOT trees. This step is represented by the right-hand green box at the bottom of Fig.\,\ref{fig:production_overview}.

Users can specify individual reactions via a web interface.
%(see Fig.~\ref{fig:production_analysis}).
Periodically, the submitted reactions are downloaded into a configuration file, which steers the analysis launch. For each reaction, the \GX~analysis library inside the JANA framework creates possible particle combinations from the reconstructed particle tracks and showers saved in the REST format. Common selection criteria are applied for exclusivity and particle identification before performing a kinematic fit, using vertex and four-momentum constraints. Displaced vertices and inclusive reactions are also supported. Objects representing successful particle combinations (e.g. $\pi^0 \rightarrow \gamma\gamma$) and other objects are managed in memory pools, and can be reused by different channels to reduce the overall memory footprint of the process. With this scheme, up to one hundred different reactions can be combined into one analysis launch processing the reconstructed data.

If the kinematic fit converged for one combination of tracks and showers, the event is stored into a reaction-specific but generic ROOT tree, made accessible to the whole collaboration. The size of the resulting ROOT trees for the full data set strongly depends on the selected reaction, but is usually small enough to be copied to the user's home institution for a more detailed analysis.

%\begin{figure}[h!]\centering
%\includegraphics[width=\textwidth]{figures/analysis_submit_form.png}
%\caption[]{\label{fig:production_analysis}Analysis submission browser form.} 
%\end{figure}
%=======================+=========================
%================  Simulation  ================
%=================================================
\section[Monte Carlo]{Monte Carlo simulation \label{sec:simulation}}
The detailed simulation of events in the Hall-D beam line and GlueX detector is performed with a Geant-based simulation.The package was originally developed within the GEANT3 framework~\cite{Brun:1987ma} and recently migrated to the GEANT4 framework~\cite{Agostinelli:2002hh,Allison:2016lfl}. The simulation framework uses the same geometry definitions and magnetic field maps as used in reconstruction and is able to simulate the primary photon beam from the bremsstrahlung radiator through the GlueX detector to the photon beam dump, as well as doing event simulations of photoproduction events in the GlueX detector itself. The simulation broadly follows the diagram as shown in Fig.~\ref{fig:MC-data-flow}. Events of interest are generated using either a predefined or user-supplied event generator. The events are read in by the Geant-based Hall-D Monte Carlo, which obtains the appropriate geometry and magnetic field information from the geometry and calibration data bases. The resulting events are then processed by a digitization package. This package obtains run-dependent information from both the calibration and run-conditions data bases and background events from a file. The simulated events have appropriate inefficiencies applied, and then together with the background events, are digitized to look like raw data from the experiment. The resulting events are then processed with same reconstruction software as used for the real events and the output is saved to a REST file. These files are then made available for physics analysis.
\begin{figure}[h!]\centering
\includegraphics[width=0.85\textwidth]{figures/MC-data-flow.pdf}
\caption[]{\label{fig:MC-data-flow}The data flow from event generator through physics analysis files for Monte Carlo events.}
\end{figure}

\subsection{Event generators \label{sec:generators}}
Simulation starts with the generation of events, which can be specific particles or reactions, or a background generator. A common tool set has been developed to minimize redundancy in code. These tools include standard methods to generate the distributions of primary photon beam energies and polarization.

One of the first generators has been used to simulate the entire photoproduction cross section and is used to study backgrounds to physics reactions as well as develop analysis tools for extracting signals. This is based on Pythia~\cite{Sjostrand:2006za}, but includes additions that describe the low-energy photoproduction cross sections. 


\subsection{HDGeant \label{sec:hdgeant}}

\subsubsection[Material thickness]{\label{sec:materialscan}Geometry definitions}


\subsection[mcsmear]{Detector Response}
\subsubsection{Signal digitization}
\subsubsection{Run-dependent effects}
\subsubsection{Background Events}

\subsection{Job submission \label{sec:jobsubmission}}

%=======================+=========================
%================  Performance  ================
%=================================================

\section[Detector performance]{Detector performance \label{sec:performance}}                                                 

The capability of the \gx{} detector in reconstructing charged and neutral particles and assembling them into fully reconstructed events has been studied in data and simulation using several photoproduction reactions.  The results of these studies are summarized in this section.

%\subsection{Charged particles \label{sec:perfcharged}} 

%\subsubsection{Efficiency \label{sec:perfchargedeff}}

%The reconstruction efficiency for different hadron species has been studied using the method described in Sec.~\ref{sec:trackeff}.   Charged pion reconstruction efficiency was studied with a sample of $\gamma p \to p \omega$, $\omega \to \pi^+\pi^-\pi^0$ events.  Charged kaon reconstruction efficiency was studied with samples of $\gamma p \to \phi p$, $\phi \to K^+K^-$ and $\gamma p \to \Lambda(1520) K^+$, $\Lambda(1520) \to p K^-$ events.  Proton reconstruction efficiency was studied with a sample of $\gamma p \to K^+ \Sigma^0$, $\Sigma^0 \to \gamma \Lambda^0$, $\Lambda^0 \to p \pi^-$ events.  
%The results are illustrated in Fig.~X. The efficiencies determined by data and simulation generally agree to within a few percent.

\subsection{Charged-particle reconstruction efficiency  \label{sec:trackeff}}
The track reconstruction efficiency was estimated by analyzing $\gamma p \rightarrow p \omega$, $\omega\rightarrow\pi^+\pi^-\pi^0$ events, where the proton, the $\pi^0$, and one of the charged pions were used to predict the three-momentum of the other charged pion. Two methods were used to calculate this efficiency, $\varepsilon=N_{found}/(N_{found}+N_{missing})$.  Events for which no track was reconstructed in the predicted region of 
phase space contributed to $N_{missing}$, while events where the expected track was reconstructed contributed to $N_{found}$.  For the first method, the $\omega$ yields for $N_{found}$ and $N_{missing}$ were estimated from the missing mass off the 
proton; for the second method, the invariant mass of the $\pi^+\pi^-\pi^0$ system was used to find $N_{found}$.  This analysis was performed for individual bins of track momentum, $\theta$, and $\phi$.
Examples of mass histograms for a typical bin in $\phi$ are shown in Fig.~\ref{fig:omega mass}.  The exercise was repeated for a sample of $\omega$ Monte Carlo events.   A comparison of the efficiency for pion reconstruction derived from the 
two methods for both Monte Carlo and experimental data is shown in Fig.~\ref{fig:tracking efficiency}.  The efficiencies for Monte Carlo and experimental data 
agree to within 5\%.

While this reaction only allows the determination of track reconstruction efficiencies for $\theta < 30^\circ$, this covers the majority of charged particles produced in \gx{} due to its fixed-target geometry.  Other reactions are being studied to determine the efficiency at larger angles.

\begin{figure}[tbp]
\begin{center}
\includegraphics[width=0.45\textwidth]{figures/MissingOmegaFit.pdf}
\includegraphics[width=0.45\textwidth]{figures/ThreePiFit.pdf}
\caption{\label{fig:omega mass}
Reconstructed mass distributions for the reaction $\gamma p \to p\pi^0\pi^{\pm}(\pi^\mp)$ for a bin in $\phi$.
  (Left) Distribution of the missing mass off the proton.
(Right) Invariant mass distribution for the $\pi^+\pi^-\pi^0$ system.  The black
curves show the results of fits to the distributions.
 (Color online)}
\end{center}
\end{figure}

\begin{figure}[tpb]
\begin{center}
\includegraphics[width=\textwidth]{figures/PiPlusEfficiency.pdf}
\includegraphics[width=\textwidth]{figures/PiMinusEfficiency.pdf}
\caption{\label{fig:tracking efficiency}
(Top row) Tracking efficiency for $\pi^+$ tracks. (Bottom row) Tracking efficiency for $\pi^-$ tracks.  (Color online)}
\end{center}
\end{figure}



\subsection{Neutral-particle efficiency\label{sec:perfneutral}}

%\subsubsection{Resolution \label{sec:perfneutralresol}}

%The invariant mass resolution of the decay $\eta \to \gamma \gamma$ has been found to primarily depend on the energy resolution of the calorimeters at \gx{}.  Therefore, in Fig.~X we show the resolution for these reconstructed decays for three classes of events: where both photons from the  $\eta \to \gamma \gamma$ are reconstructed in the BCAL, were both are reconstructed in the FCAL, and where one photon is reconstructed in the BCAL and one in the FCAL.

%\subsubsection{Efficiency \label{sec:perfneutraleff}}

Photon reconstruction efficiency has been studied using different methods for the FCAL and BCAL.  In the FCAL, absolute photon reconstruction efficiencies have been determined using the ``tag-and-probe'' method with a sample of photons from the reaction $\gamma p \to \omega p$, $\omega \to \pi^+\pi^-\pi^0$, $\pi^0 \to \gamma (\gamma)$, where one final photon is allowed but not required to be reconstructed.  The yields with and without the reconstructed photon are determined using two methods.  In the first method, the $\omega$ yield is determined from the missing-mass spectrum, $M_X(\gamma p \rightarrow pX)$, selecting on whether only one or both reconstructed photons are consistent with a final-state $\pi^0$. In the second method,  the count when both photons are found is determined from the $\omega$ yield from the fully reconstructed invariant mass $M(\pi^+\pi^-\gamma\gamma)$. If the photon is not reconstructed, the $\omega$ yield is determined by a fit to the distribution of the missing mass off the proton.  Both methods yield consistent results, with a reconstruction efficiency generally above 90\%, and within 5\% or less agree with the efficiencies determined from simulation.

\begin{figure}[tbp]
\begin{center}
\includegraphics[width=0.45\textwidth]{figures/OmegaCompareE.pdf}
\includegraphics[width=0.45\textwidth]{figures/OmegaCompareTheta.pdf}
\caption{\label{fig:fcalphotoneff}
Photon reconstruction efficiency in FCAL determined from $\gamma p \to \omega p$, $\omega \to \pi^+\pi^-\pi^0$, $\pi^0 \to \gamma (\gamma)$ as a function of (left) photon energy and (right) photon polar angle.  Good agreement between data and simulation is observed in the fiducial region $\theta = 2^\circ - 10.6^\circ$. (Color online)
}
\end{center}
\end{figure}

A relative photon efficiency determination has been performed using $\pi^0\to\gamma\gamma$ decays, which spans the full angular range detected in \gx{}.  A sample of fully reconstructed $\gamma p \to  \pi^+\pi^-\pi^0 p$ events were inspected, taking advantage of the $\pi^0\to\gamma\gamma$ decay isotropy in the center-of-mass frame.  Thus, any anisotropy indicates an inefficiency in the detector. Results from this analysis are illustrated in Fig.~\ref{fig:bcalpi0photoneff}. Generally, this relative efficiency is above 90\%, and agrees within 5\% of that determined from simulation.  

The models for the simulated response of both calorimeters are being updated, and the final agreement between photon efficiency determined in data and simulation is expected to improve.

\begin{figure}[tbp]
\begin{center}
\includegraphics[width=\textwidth]{figures/plot_CostheEff_NIM_jun19.pdf}
\caption{\label{fig:bcalpi0photoneff}
Ratios of relative photon reconstruction efficiency between data and simulation determined from $\pi^0\to\gamma \gamma$ decays in $\gamma p \to  \pi^+\pi^-\pi^0 p$ events.  The efficiency ratios are shown for the cases where (left) both photons were measured in the BCAL, (middle) both photons were measured in the FCAL, and (right) one photon was measured in the BCAL and the other in the FCAL.
}
\end{center}
\end{figure}


Detailed studies of detector performance determined the standard fiducial region for most analyses to be $\theta = 2^\circ - 10.6^\circ$ and $\theta > 11.3^\circ$.  These requirements avoid the region dominated by beam-related backgrounds at small $\theta$ and the transition region between the BCAL and FCAL, where shower reconstruction is difficult.

\subsection{Kinematic fitting \label{sec:perffitting}}

Kinematic fitting is a powerful tool to improve the resolution of measured data and to distinguish between different reactions.  In \gx{}, this method takes advantage of the fact that the initial state is very well known, with the target proton at rest, and the incident photon energy measured with very high precision ($<0.1\%$). This knowledge of the initial state gives substantial improvements in the kinematic quantities determined for exclusive reactions.  The most common kinematic fits that are performed are those that impose energy-momentum conservation between the initial and final-state particles.  Additional optional constraints in these fits are for the four-momenta of the daughters of an intermediate particle to add up to a fixed invariant mass, and for all the particles to come from a common vertex (or multiple vertices, in the case of reactions containing long-lived, decaying particles).

To illustrate the performance of the kinematic fit, we use a sample of $\gamma p \to \eta p$, $\eta \to \pi^+\pi^-\pi^0$ events selected using a combination of standard particle identification and simple kinematic selections.  
%Fig.~\ref{fig:etakinfit} shows the $\eta\to \pi^+\pi^-\pi^0$ peak and the improvement in $\pi^+\pi^-\pi^0$ mass resolution using the kinematic fit from 2.6~MeV to 1.7~MeV, which is typical of low-multiplicity meson production reactions.  
The use of the kinematic fit improves the $\eta$-mass resolution  from  2.6~MeV to 1.7~MeV, which is typical of low-multiplicity meson production reactions.  
The quality of the kinematic fit is determined using either the probability calculated from the $\chi^2$ of the fit and the number of degrees-of-freedom or the $\chi^2$ of the fit itself. 
The distributions of the kinematic fit $\chi^2$ and probability are illustrated in Fig.~\ref{fig:kinfitperform} for both reconstructed and simulated data.  The agreement between the two distributions is good for small $\chi^2$ (large probability), and flat over most of the probability range, indicating good overall performance for most signal events.  The disagreement between the two distributions at larger $\chi^2$ (probability $<0.2$) is due to a combination of background events and deficiencies in the modelling of poorly measured events with large resolution.

The performance of the reconstruction algorithms and kinematic fit can be studied through investigating the ``pull'' distributions, where the pull of a variable $x$ is defined by comparing its measured values and uncertainties and those resulting from the kinematic fit as
\begin{equation}
    \text{pull}_x = \frac{x_\text{fitted} - x_\text{measured}}{\sqrt{\sigma_{x,\text{measured}}^2 - \sigma_{x,\text{fitted}}^2}}.
\end{equation}
If the parameters and covariances of reconstructed particles are Gaussian, are measured accurately, and the fit is performing correctly, then these pull values are expected to have a Gaussian distribution centered at zero with a width $\sigma$ of 1.  If the pull distributions are not centered at zero, this is an indication that there is a bias in the measurements or the fit.  If $\sigma$ varies from unity, this is an indication that the covariance matrix elements are not correctly estimated.  

As an example, the pull distributions for the momentum components of the $\pi^-$ in reconstructed $\gamma p \to \eta p$, $\eta \to \pi^+\pi^-\pi^0$ events are shown in Fig.~\ref{fig:kinfitpulls}.  Both real and simulated data have roughly Gaussian shapes with similar widths.  More insight into the stability of the results of the kinematic fit can be found by studying the variation of the means and widths of the fit distributions as a function of the fit probability.  The results of such a study are summarized in Fig.~\ref{fig:kinfitstudy}, where broad agreement between the results from real and simulated data is seen.  The means of the pull distributions are generally around zero (with $p_x$ and its mean of roughly $-0.1$ a notable exception), and the widths within about 20\% of unity.  This level of performance and agreement between data and simulation is acceptable for the initial analysis of data, where very loose cuts on the kinematic fit $\chi^2$ are performed, and steady improvement in the modeling of the covariance matrices of reconstructed particles is expected to continue.


\begin{figure}[tbp]
\begin{center}
\includegraphics[width=0.75\textwidth]{figures/gluex_nim_kfit_prob.pdf}
\caption{\label{fig:kinfitperform}
Distribution of kinematic fit (left) probability and (right) $\chi^2$ for reconstructed $\gamma p \to \eta p$,  $\eta \to \pi^+\pi^-\pi^0$ events in data and simulation.  Both distributions agree reasonably for well-measured events, and diverge due to additional background in data and differences in modeling poorly-measured events.
 (Color online)}
\end{center}
\end{figure}

\begin{figure}[tbp]
\begin{center}          \includegraphics[width=0.29\textwidth]{figures/gluex_nim_PiMinus_PxPull.pdf}
\includegraphics[width=0.29\textwidth]{figures/gluex_nim_PiMinus_PyPull.pdf}
\includegraphics[width=0.29\textwidth]{figures/gluex_nim_PiMinus_PzPull.pdf}

\caption{\label{fig:kinfitpulls}
Pull distributions for momentum components of the $\pi^-$ from reconstructed $\gamma p \to \eta p$,  $\eta \to \pi^+\pi^-\pi^0$ events in data and simulation for events with fit probability $>0.01$: (left) $p_x$, (center) $p_y$, (right) $p_z$.
 (Color online)}
\end{center}
\end{figure}

\begin{figure}[tbp]
\begin{center}          \includegraphics[width=0.3\textwidth]{figures/gluex_nim_pullspx_pulls_mean_data.pdf}
\includegraphics[width=0.3\textwidth]{figures/gluex_nim_pullspy_pulls_mean_data.pdf}
\includegraphics[width=0.3\textwidth]{figures/gluex_nim_pullspz_pulls_mean_data.pdf}

\includegraphics[width=0.3\textwidth]{figures/gluex_nim_pullspx_pulls_mean_mc.pdf}
\includegraphics[width=0.3\textwidth]{figures/gluex_nim_pullspy_pulls_mean_mc.pdf}
\includegraphics[width=0.3\textwidth]{figures/gluex_nim_pullspz_pulls_mean_mc.pdf}

\includegraphics[width=0.3\textwidth]{figures/gluex_nim_pullspx_pulls_sigma_data.pdf}
\includegraphics[width=0.3\textwidth]{figures/gluex_nim_pullspy_pulls_sigma_data.pdf}
\includegraphics[width=0.3\textwidth]{figures/gluex_nim_pullspz_pulls_sigma_data.pdf}

\includegraphics[width=0.3\textwidth]{figures/gluex_nim_pullspx_pulls_sigma_mc.pdf}
\includegraphics[width=0.3\textwidth]{figures/gluex_nim_pullspy_pulls_sigma_mc.pdf}
\includegraphics[width=0.3\textwidth]{figures/gluex_nim_pullspz_pulls_sigma_mc.pdf}

\caption{\label{fig:kinfitstudy}
Pull means (top) and sigmas (bottom) for the momentum components of each particle as a function of the minimum probability required of the fit from reconstructed $\gamma p \to \eta p$,  $\eta \to \pi^+\pi^-\pi^0$ events.
 (Color online)}
\end{center}
\end{figure}

\subsection{Invariant-mass resolution \label{sec:perfchargedresol}}

The invariant-mass resolution for resonances depends on the momenta and angles of their decay products.  This resolution has been studied using several different channels, which are illustrated in Figs.~\ref{fig:invmass1} and \ref{fig:invmass2}. A typical meson production channel including both charged particles and photons, $\omega \to \pi^+\pi^-\pi^0$ from $\gamma p \to \omega p$, is shown in the left panel of Fig.~\ref{fig:invmass1}. The distribution shows the strong peak due to $\omega$ meson production.  Other structures are also seen, such as peaks corresponding to the production of $\eta$ and $\phi$ mesons.  The $\omega$ peak resolution obtained is 26.1~MeV when using only the reconstructed  particle 4-vectors, and improves to 16.4~MeV after a kinematic fit. The invariant-mass distribution of $\pi^+\pi^-$ from $\gamma p \to K_S K^+ \pi^- p$, $K_S\to\pi^+\pi^-$ exhibits the peak due to $K_S\to\pi^+\pi^-$ decays (right panel of Fig.\,\ref{fig:invmass1}).  The $K_S$ peak resolution is 17.0~MeV using only the reconstructed charged particle 4-vectors, and improves to  8.6~MeV after a kinematic fit imposing energy and momentum conservation. The dependence of the $K_S\to\pi^+\pi^-$ invariant-mass resolution as a function of $K_S$ momentum is shown in Fig.\,\ref{fig:invmass1a} , both before and after an energy/momentum-constraint kinematic fit.  

The invariant mass of $\Lambda^0\pi^-$ from $\gamma p \to K^+ K^+ \pi^- \pi^- p$ is shown in the left panel of Fig.\,\ref{fig:invmass2},  illustrating the peak due to $\Xi^- \to \pi^- \Lambda^0$, $\Lambda^0 \to p \pi^-$.  The $\Xi^-$ peak resolution obtained is 7.3~MeV when using only the reconstructed charged particle 4-vectors, and improves to 4.6~MeV after a kinematic fit imposing energy and momentum conservation and the additional constraint that the mass of the $p \pi^-$ pairs must be that of the $\Lambda^0$ mass.  The $e^+e^-$ invariant mass distribution from kinematically fit $\gamma p \to e^+e^- p$ events is shown in the right panel of Fig.\,\ref{fig:invmass2}, illustrating the peak due to $J/\psi\to e^+e^-$.  The resolution of the peak is 13.7~MeV.        

\begin{figure}[tpb]
\begin{center}
\includegraphics[width=0.45\textwidth]{figures/omega_inv_mass_probCut_001.pdf}
\includegraphics[width=0.4\textwidth]{figures/kskpi_mass_spect.pdf}
\caption{\label{fig:invmass1}
(Left top) $\pi^+\pi^-\pi^0$ invariant-mass distribution from $\gamma p \to \pi^+\pi^-\pi^0 p$ (Right top) $\pi^+\pi^-$ invariant mass distribution from $\gamma p \to K_S K^+ \pi^- p$, $K_S\to\pi^+\pi^-$. (Color online)}
\end{center}
\end{figure}


\begin{figure}[tpb]
\begin{center}\includegraphics[width=0.5\textwidth]{figures/kskpi_mass_resol.pdf}
\caption{\label{fig:invmass1a}
$K_S\to\pi^+\pi^-$ invariant mass resolution for the events shown in Fig.\,\ref{fig:invmass1}, as a function of $K_S$ momentum, both before and after a kinetic fit, which constrains energy and momentum conservation.  
(Color online)}
\end{center}
\end{figure}


\begin{figure}[tpb]
\begin{center}
\includegraphics[width=0.42\textwidth]{figures/XimMass_2017-ver30.pdf}
\includegraphics[width=0.42\textwidth]{figures/jpsi_mass.pdf}
\caption{\label{fig:invmass2}
(Left) $\Lambda^0\pi^-$ invariant mass distribution from $\gamma p \to K^+ K^+ \pi^- \pi^- p$. (Right) $e^+e^-$ invariant mass distribution from kinematically fit $\gamma p \to e^+e^- p$ events. (Color online)}
\end{center}
\end{figure}

\subsection{Particle identification \label{sec:perfpid}}

Particle identification in \gx{} uses information from both energy loss in different detector systems and time-of-flight measurements.  This information can be used for identification in several ways.  The simplest method is to apply selections directly on the relevant PID variables.  To include detector resolution information, one can create a $\chi^2$ variable comparing a measured value to the expected value for a particular hypothesis, that is
\begin{equation}
    \chi^2(p) = \left(  \frac{ X(\mathrm{measured}) - X(\mathrm{expected})_p}{\sigma_X} \right)^2
\end{equation}
where $X$ is the given PID variable, $p$ is the particle hypothesis, and $\sigma_X$ is the resolution of this variable.  Multiple PID variables can be combined into one probability, or a figure-of-merit.   Standard, loose selections on time-of-flight and energy loss are sufficient for initial physics analyses, while the performance of more complicated selections is being actively studied.

At sufficiently large $\theta$, the energy loss for charged particles in the central drift chamber $dE/dx$ can be used.   Fig.~\ref{fig:performcdcdedx} illustrates these distributions for positively charged particles, showing a clear separation of pions and protons in the momentum range $\lesssim 1$~GeV. % along with the standard selection used to separate pions and protons.
The $dE/dx$ resolution is approximately 27\%, with the separation between the pion and proton bands dropping from about $8\sigma$ at $p=0.5$~GeV/$c$ to about $2\sigma$ at $p=1.0$~GeV/$c$, with both bands fully merged by $p=1.5$~GeV/$c$.

\begin{figure}[tbp]
\begin{center}
%\includegraphics[width=0.6\textwidth]{figures/cdc_pos_dedx.pdf}
\includegraphics[width=0.6\textwidth]{figures/cdc_dedx.pdf}
\caption{\label{fig:performcdcdedx}
CDC energy loss ($dE/dx$) for positively charged particles that have at least 20 hits in the detector, as a function of measured particle momentum.  The band corresponding to protons curves upwards, showing a larger energy loss than pions and other lighter particles at low momentum.  The two bands show a clear separation for momenta  $\lesssim 1$~GeV.  A faint kaon band can be seen between them.
}
\end{center}
\end{figure}

The primary means of particle identification is through time-of-flight measurements, and information from several sources is combined to make the most accurate determination.  The RF reference signal from the accelerator is used to define the time when each photon bunch enters the target.  The reconstructed final-state particles are used to determine which photon bunch most likely generated the detected reaction, with the primary determination coming from the signals from the Start Counter associated with the charged particle tracks.  The photon bunch determination has a resolution of $<10$~ps. Each charged particle is associated with additional timing information based on the hit in the highest resolution detector (for example the BCAL or TOF).  The flight  time to this measured hit $t_\mathrm{meas}$ relative to the time of the photon bunch that generated the event $t_\mathrm{RF}$ can be used to distinguish between particles of different mass.  Two common variables that are used are the velocity ($\beta$) determined using the measured time-of-flight and the momentum of the particle, and $\Delta t_\mathrm{RF}$, the difference between the measured and RF times after they both have been extrapolated back to the center of the target, assuming some particle-mass hypothesis.
An example of the separation between different particle types can be seen in Fig.~\ref{fig:betavsp}.
The loose selections used for initial analyses of this data placed on the $\Delta t_\mathrm{RF}$ distributions and the momentum dependence of the resolution of this variable in different detectors are shown in Fig.~\ref{fig:timingresol}.  
Requiring reconstructed particles to have  $\Delta t_\mathrm{RF} \lesssim 1-2$~ns has been found to be sufficient for analyses of high-yield channels which are the focus of initial analysis.  The study of the selections required for more demanding channels is ongoing.

\begin{figure}[tbp]
\begin{center}          
\includegraphics[width=0.29\textwidth]{figures/bcal_deltat_resol.pdf}
\includegraphics[width=0.29\textwidth]{figures/fcal_deltat_resol.pdf}
\includegraphics[width=0.29\textwidth]{figures/tof_deltat_resol.pdf}

\caption{\label{fig:timingresol}
Resolution as a function of particle momentum for  $\Delta t_\mathrm{RF}$ in various subdetectors: (left) BCAL, (center) FCAL, (right) TOF
 (Color online)}
\end{center}
\end{figure}


Electrons are identified using the ratio of their energy loss in the electromagnetic calorimeters $E$ to the momentum reconstructed in the drift chambers $p$.  This $E/p$ ratio should be approximately unity for electrons and less for hadrons.  The overall distribution of this variable is illustrated for both calorimeters in  Fig.~\ref{fig:performeop}.  Other variables, such as the shape of the showers generated by the charged particles in the calorimeter, promise to provide additional information to separate electron and hadron showers.

\begin{figure}[tbp]
\begin{center}
\includegraphics[width=0.4\textwidth]{figures/fcal_ep.pdf}
\includegraphics[width=0.4\textwidth]{figures/fcal_ep.pdf}
\caption{\label{fig:performeop}
Electron identification in the calorimeters is performed using the $E/p$ variable, the ratio of the energy loss in the electromagnetic calorimeters ($E$) to the momentum reconstructed in the drift chambers ($p$).  This distribution is shown for selected samples of electrons from (left) $\gamma p \to \pi^0$, $\pi^0 \to e^+e^-\gamma$, where the $e^\pm$ are reconstructed in the FCAL, and (right) ....
}
\end{center}
\end{figure}


%Particle identification and mis-identification rates are summarized in Fig.~X.

%\subsection{Systematic uncertainties \label{sec:systematics}}

%=======================+==============================
%============    Summary   =============
%======================================================

\section{Summary and outlook\label{sec:summary} }
We have presented the design, construction and performance during the first phase of operation of the  \gx~ experiment in Hall D at Jefferson Lab. 
The experiment operated routinely at an incident photon flux of $2\times 10^{7}$ photons/s in the coherent peak\footnote{Defined as 0.6 GeV below the coherent edge, which varied somewhat depending on the primary incident electron beam energy.} with an open trigger taking data
at 40 kHz and recording 600 MB/s to tape and live time $>$95\%. 
During this time the experiment accumulated  121.4 pb$^{-1}$ in the coherent peak. and 319.4 pb$^{-1}$ for $E_\gamma>$8.1 GeV. We accumulated approximately 270 billion triggers during this period. as shown in Fig.\,\ref{fig:plot_rcdb3_phaseI}.  

\begin{figure}[tbh]\centering
\includegraphics[width=0.48\textwidth]{figures/plot_rcdb3_phaseI.eps}
\caption{\label{fig:plot_rcdb3_phaseI} 
Plot of integrated number of triggers versus the number of live days
in 2016, 2017 and 2018. The triggers of the four diamond configurations fall on top of one another, as we attempted to match the amount of data taken for each configuration. 
(Color online)
 }
\end{figure}   

We have verified the operational characteristics of the charged and neutral particle detector systems and checked that individual systems performed as designed. We have also demonstrated that the detector as a whole is able to reconstruct exclusive final states, determined the reconstruction efficiencies and validated our Monte Carlo simulation against data. The infrastructure is in place to process our high volume of data both on the JLab computing farm as well on other offsite facilities. The use of these tools gives us the ability to process the data in a timely fashion.

Future running will include taking data at higher luminosity  and with improved particle identification capability. 
The \gx~experiment has already implemented the necessary infrastructure to allow us to operate at a flux of $5\times10^{7}$ photons/s in the coherent peak for the upcoming run periods and has added a new DIRC detector\footnote{Four ``bar boxes" from the BaBar \cite{Aubert:2001tu} detector have been installed and tested.} to extend the particle identification of kaons to higher momenta. 

   

%================+===================
%========  Acknowledgments  =========
%====================================

\section{Acknowledgments}  
We gratefully acknowledge the outstanding efforts of technical support at all the collaborating institutions and the support groups at Jefferson Lab that completed the assembly, installation,
and maintenance of the detector. We acknowledge the contributions of D. Bennett, M. Lara, A. Subedi and P. Smith to the construction and commissioning of the Forward Calorimeter. This work was supported in part by the U.S. Department of Energy, the U.S. National Science Foundation, the German Research Foundation, Forschungszentrum J\"ulich GmbH, GSI Helmholtzzentrum f\"{u}r Schwerionenforschung GmbH, the Russian Foundation for Basic Research, the UK Science and Technology Facilities Council, the Chilean Comisi\'{o}n Nacional de Investigaci\'{o}n Cient\'{i}fica y Tecnol\'{o}gica, the National Natural Science Foundation of China, and the China Scholarship Council. This material is based upon work supported by the U.S. Department of Energy, Office of Science, Office of Nuclear Physics under contract DE-AC05-06OR23177. 

\newpage

\section*{References}
   
%\nocite{*}
\bibliography{GlueX_nim}
%\bibliographystyle{unsrt}
%\bibliographystyle{elsarticle-num}
\bibliographystyle{elsarticle-num-modified}

\end{document}
